%%%%%%%%%%%%%%%%%%%%%%%%%%%%%%%%%%%%%%%%%%%%%%%%%%%%%%%%%%%%%%%%%
% Dissertacao de Mestrado / Dept Fisica, CFM, UFSC              %
% Lacerda@UFSC - 2013                                           %
%%%%%%%%%%%%%%%%%%%%%%%%%%%%%%%%%%%%%%%%%%%%%%%%%%%%%%%%%%%%%%%%%

%:::::::::::::::::::::::::::::::::::::::::::::::::::::::::::::::%
%                                                               %
%                          Abstract                             %
%                                                               %
%:::::::::::::::::::::::::::::::::::::::::::::::::::::::::::::::%

\begin{abstract}

Spectroscopic surveys of galaxies, like the SDSS, are currently being taken to a next level with integral field units,
turning the focus from the properties of galaxies as a whole to the internal physics of galaxies. The CALIFA survey is a
pioneer in this new generation, providing spatially resolved spectroscopy for hundreds of galaxies of all shapes and
masses. Several studies have been conduted with datacubes from this survey, examining the spatial distributions of
emission lines, stellar populations, gaseous and stellar kinematics, etc. In this explotatory work we approach the
analysis of CALIFA datacubes from the mathematical perspective of principal component analysis (PCA) tomography
techniques developed by Steiner and collaborators. We apply the PCA tomography to 9(?8?) CALIFA galaxies of early and
late type galaxies, as well as a couple or merging systems. Emission lines are masked in our analysis to concentrate on
the stellar population properties.

The physical and angular scales spaned by our datacubes (which cover whole galaxies) are much larger than those spanned
by previous applications of this technique. This difference prompted us to introduce a simple variation on the pre-PCA
stage, normalizing  all  spectra in a datacube to a common flux scale, which highlight qualitative (as opposed to
amplitude-related quantitative) spectral variance in the datacube. We also apply PCA to the synthetic spectra obtained
with \starlight\ fits, and to the $\log$ of the spectra. The first eliminates noise and flux-calibration effects on the
data, while the latter may be suitable to investigate objects with large spatial variations in the dust content. A well
known caveat with PCA is that it does not assign physical meaning to its results. To tackle this problem we use a
\starlight-based stellar population analysis of the same data to reverse-engineer the astrophysical meaning of the
principal components.

We find that \ojo  AQUI VAO ALGUNS RESULTADOS GERAIS (tipo .... kinematics introduces variance which we do not want/like
...), E DEPOIS ALGUNS COMENTARIOS SOBRE CASOS INDIVIDUAIS, COMO: In the case of the merging systems Arp 220 (CALIFA
0802) and NGC ??? (CALIFA 213 ?273?), the PCA of the $\log$ flux blablabla.

\end{abstract}

% End of abstract



