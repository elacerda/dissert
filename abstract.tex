%%%%%%%%%%%%%%%%%%%%%%%%%%%%%%%%%%%%%%%%%%%%%%%%%%%%%%%%%%%%%%%%%
% Dissertacao de Mestrado / Dept Fisica, CFM, UFSC              %
% Lacerda@UFSC - 2013                                           %
%%%%%%%%%%%%%%%%%%%%%%%%%%%%%%%%%%%%%%%%%%%%%%%%%%%%%%%%%%%%%%%%%

%:::::::::::::::::::::::::::::::::::::::::::::::::::::::::::::::%
%                                                               %
%                          Abstract                             %
%                                                               %
%:::::::::::::::::::::::::::::::::::::::::::::::::::::::::::::::%

\begin{abstract}

Spectroscopic surveys of galaxies, like the SDSS, are currently being taken to a next level with integral field units,
turning the focus from the properties of galaxies as a whole to the internal physics of galaxies. The CALIFA survey is a
pioneer in this new generation, providing spatially resolved spectroscopy for hundreds of galaxies of all shapes and
masses. Several studies have been conducted with datacubes from this survey, examining the spatial distributions of
emission lines, stellar populations, gaseous and stellar kinematics, etc. In this explotatory work we approach the
analysis of CALIFA datacubes from the mathematical perspective of principal component analysis (PCA) tomography
techniques developed by Steiner and collaborators. We apply the PCA tomography to 8 CALIFA galaxies. (4 {\em
late-type}, 2 {\em early-type} and 2 {\em mergers}). Emission lines are masked in our analysis to concentrate on the
stellar population properties.

The physical and angular scales spaned by our datacubes (which cover whole galaxies) are much larger than those spanned
by previous applications of this technique. This difference prompted us to introduce a simple variation on the pre-PCA
stage, normalizing  all  spectra in a datacube to a common flux scale, which highlight qualitative (as opposed to
amplitude-related quantitative) spectral variance in the datacube. This eliminates noise and flux-calibration effects on
the data. A well known caveat with PCA is that it does not assign physical meaning to its results. To tackle this
problem we use a \starlight-based stellar population analysis of the same data to reverse-engineer the astrophysical
meaning of the principal components.

In every galaxy we are able to find some correlation between principal components and physical parameters. We find that
the first principal component in spiral galaxies usually correlates with the mean stellar age (\meanL{\log\ t}). In {\em
mergers} the first component reflects essencially variations of the global extinction parameter ($A_V$). In some cases,
the PCA tomography helps us find errors in the original datacube because usually the huge variance imposed by damaged
pixels appears in isolated principal components.

\end{abstract}

% End of abstract



