%%%%%%%%%%%%%%%%%%%%%%%%%%%%%%%%%%%%%%%%%%%%%%%%%%%%%%%%%%%%%%%%%
% Dissertacao de Mestrado / Dept Fisica, CFM, UFSC              %
% Lacerda@UFSC - 2013                                           %
%%%%%%%%%%%%%%%%%%%%%%%%%%%%%%%%%%%%%%%%%%%%%%%%%%%%%%%%%%%%%%%%%

%:::::::::::::::::::::::::::::::::::::::::::::::::::::::::::::::%
%                                                               %
%                          Capítulo 3                           %
%                                                               %
%:::::::::::::::::::::::::::::::::::::::::::::::::::::::::::::::%

%***************************************************************%
%                                                               %
%                      PCA e Tomografia PCA                     %
%                                                               %
%***************************************************************%

\chapter{PCA e Tomografia PCA}
\label{sec:PCAeTomoPCA}

De medidas fisiológicas, como pulsação e respiração, até reconhecimento de
padrões em sistemas complexos como reconhecimento facial e criptografia,
passando por compactação de imagem, neurosciência, redução de ruídos em sinal
gerado por transdures de qualquer espécie (óticos, acústicos, sísmicos), podemos ver
atuação de técnicas de PCA. \ojo Brincando com a própria técnica, o PCA em si é
uma CP de um universo matemático-estatístico.

%***************************************************************% 
%                                                               %
%                              PCA                              %
%                                                               % 
%***************************************************************%

\section{Principal Component Analisys}
\label{sec:PCAeTomoPCA:PCA}

Baseada em encontrar os eixos com maiores variâncias em um conjunto de variáveis
(no nosso caso, fluxos por lambda e por zona), a técnica PCA vem sendo de grande
utilidade quando o assunto é estatística com muitas variáveis. Através de
operações relativamente simples computacionalmente usando álgebra linear, é
feito uma rotação na base de dados original, gerando uma nova base ortonormal
não correlacionada através de um conjunto de autovalores e autovetores.

Existem diversas formas de se calcular essa base final. A prova matemática que
você pode obter essa base é feita através de multiplicadores de Lagrange,
calculando os autovetores e autovalores ($\mathbf{e}{}_k$ e $\lambda_k$) que
maximizam o valor de $\mathbf{e}{}_k^T \cdot \mathbf{C}{}_{cov} \cdot
\mathbf{e}{}_k$ (ver Eq. \ref{eq:PCA:covMatrix}) sujeito a restrição de que um
autovetor deve ser ortogonal a qualquer outro da base ($\mathbf{e}{}_i^T
\mathbf{e}{}_j = 0$) e que deve todos devem ser normalizados ($\mathbf{e}{}_i^T
\mathbf{e}{}_i = 1$) \citep[][cap 1]{JolliffePCA1986}. Para esse cálculo usamos
a biblioteca científica \texttt{SciPy}\footnote{\url{http://scipy.org/}}
(\ref{fig:PCA:covMatrix}), que encontra os autovetores e autovalores da matriz
de correlação.

\subsection{PCA das galáxias do CALIFA}

Conforme a Seção \ref{sec:CALePyC:PyCASSO} vimos que os dados das galáxias estão
acessíveis no PyCASSO separados por zonas. Então, por exemplo, os espectros
estão armazenados em forma de uma matriz $(n \times m)$ com $n$ zonas e $m$
comprimentos de onda.

\begin{equation}
    \label{eq:PCA:fluxMatrix}
    \textbf{F}{}_{z \lambda} = \left[
    \begin{array}{ccccc}
        f_{z_0 \lambda_0} & f_{z_0 \lambda_1} & f_{z_0 \lambda_2} & ... & f_{z_0 \lambda_m} \\
        f_{z_1 \lambda_0} & f_{z_1 \lambda_1} & f_{z_1 \lambda_2} & ... & f_{z_1 \lambda_m} \\
        f_{z_2 \lambda_0} & f_{z_2 \lambda_1} & f_{z_2 \lambda_2} & ... & f_{z_2 \lambda_m} \\
        ...               & ...               & ...               & ... & ...               \\
        f_{z_n \lambda_0} & f_{z_n \lambda_1} & f_{z_n \lambda_2} & ... & f_{z_n \lambda_m} 
    \end{array} 
    \right]
\end{equation}

Calculamos então o espectro médio de uma galáxia através da equação $\langle
\textbf{F}{}_\lambda \rangle = (1 / n) \sum_{i=0}^{n} f_{z_i}{}_{\lambda}$ e
então subtraímos a média de todos os espectros ($\textbf{I}{}_{z \lambda} =
\textbf{F}{}_{z \lambda} - \langle \textbf{F}{}_\lambda \rangle$) para o cálculo
da matriz de covariâncias usando um conjunto de dados com média zero. Vemos que
a matriz de covariância possui dimensão $\lambda \times \lambda$. 

\begin{equation}
	\label{eq:PCA:covMatrix}
	\mathbf{C}{}_{cov} = \frac{[\mathbf{I}{}_{z \lambda}]^T \cdot \mathbf{I}{}_{z
	\lambda}}{n - 1}
\end{equation}

Agora calculamos os autovalores e autovetores da matriz de covariância. Neste
trabalho usamos autoespectros ao inves de autovetores pois são autovetores de
uma matriz de covariâncias entre espectros de cada zona. Então ordenamos os
autovetores decrescentemente pelo valor de seus autovalores. Os autoespectros
são as CPs e os autovalores as variâncias. Isso feito, temos então o que
necessitamos para iniciar o cálculo do Tomograma PCA.

\begin{figure}
\begin{python}
# Carregar arquivo FITS com os dados.
from pycasso import fitsQ3DataCube
K = fitsQ3DataCube('K0277_synthesis_suffix.fits')

# Calcular o espectro medio de uma galaxia. 
# K.f_obs tem dimensao (lambda, zona), portanto, 
# fazemos o espectro medio de todas as zonas.
f_obs_mean__l = K.f_obs.mean(axis = 1)

# Subtraimos a media
I_obs__zl = K.f_obs.transpose() - f_obs_mean__l

# Calcular a matrix de convariancia
import scipy as sp
n = K.N_zone
dot_product = sp.dot(I_obs__zl.transpose(), I_obs__zl)
covMat__ll = dot_product / (n - 1.0)   

# Calcular os autovalores e autovetores
w, e = linalg.eigh(covMat__ll)

# Ordenar os autovetores decrescentemente pelo seu autovalor
S = sp.argsort(w)[::-1]
eigval = W[S]
eigvect = e[:, S]
 
\end{python}
	\caption[Exemplo de cálculo de PCA usando o PyCASSO e SciPy] {Cálculo do
	procedimento completo de PCA para os espectros observados de uma galáxia do
	CALIFA usando o PyCASSO e a biblioteca científica de Python chamada
	\texttt{SciPy}. No final do código temos eigval e eigvect que são os
	autovalores e autovetores ordenados em forma decrescente.}
	\label{fig:PCA:covMatrix}
\end{figure}

Muitas figuras de CPs e suas utilizações serão mostradas no Capítulo
\ref{sec:result} juntamente com a Tomografia PCA e as comparações com os
parâmetros físicos da sintese de populações estelar.

%***************************************************************%
%                                                               %
%                        Tomografia PCA                         %
%                                                               %
%***************************************************************%

\section{Tomografia PCA}
\label{sec:PCAeTomoPCA:TomoPCA}

Na técnica de PCA procuramos os autoespectros (CPs) da matriz de correlação,
ordenados pela variância, formam uma base ($\mathbf{E}{}_{\lambda k}$) onde 
podemos projetar os nossos dados através da transformação:

\begin{equation}
	\label{eq:TomoPCA:tomogram2D}
	\mathbf{T}{}_{z k} = \mathbf{I}{}_{z \lambda} \cdot \mathbf{E}{}_{\lambda k}
\end{equation}

Projetamos nossa matriz de observáveis com a média subtraida ($\mathbf{I}{}_{z
\lambda}$) na base das CPs ($\mathbf{E}{}_{\lambda k}$). Na posse dessa nova
matriz transformada e de um mapa que leve de zona para uma par de coordenada ($z
\to (x, y)$), podemos montar assim uma imagem. Cada imagem funciona como uma
``fatia'' de um cubo de dados expandido na nova base, assim formando a
Tomografia PCA\footnote{\url{http://www.astro.iag.usp.br/~pcatomography/}},
criada e assim batizada por \citet{Steiner2009}, que em seu artigo faz um
paralelo com fatias de um espaço tridimensional (tomograma do corpo humano, por
exemplo) ou no espaço de velocidades (Tomografia Doppler). Cada ``fatia'' possui
um autoespectro relacionado que, em conjunto, trazem novas perspectivas e ideias
para a interpretação de ambos. A passagem de coordenadas $z \to (x, y)$ está
exemplificada na Figura \ref{fig:mapaIdade} através da função de
\texttt{zoneToYX} dentro do PyCASSO.


Com isso em mãos podemos gerar nosso próprios tomogramas nas galáxias do CALIFA.


Assim formando 
\ojo Aqui vão as fórmulas, exemplos e referências da Tomografia PCA.

%% End of this chapter
