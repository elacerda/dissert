%%%%%%%%%%%%%%%%%%%%%%%%%%%%%%%%%%%%%%%%%%%%%%%%%%%%%%%%%%%%%%%%%
% Dissertacao de Mestrado / Dept Fisica, CFM, UFSC              %
% Lacerda@UFSC - 2013                                           %
%%%%%%%%%%%%%%%%%%%%%%%%%%%%%%%%%%%%%%%%%%%%%%%%%%%%%%%%%%%%%%%%%

%:::::::::::::::::::::::::::::::::::::::::::::::::::::::::::::::%
%                                                               %
%                          Capítulo 3                           %
%                                                               %
%:::::::::::::::::::::::::::::::::::::::::::::::::::::::::::::::%

%***************************************************************%
%                                                               %
%                      PCA e Tomografia PCA                     %
%                                                               %
%***************************************************************%

\chapter{PCA e Tomografia PCA}
\label{sec:PCAeTomoPCA}

De medidas fisiológicas, como pulsação e respiração, até reconhecimento de
padrões em sistemas complexos como reconhecimento facial e criptografia,
passando por compactação de imagem, neurosciência e redução de ruídos em dados,
podemos ver atuação de técnicas de PCA. Neste capítulo revisamos os fundamentos matemáticos do PCA e de sua versao para cubos de dados (Tomografia PCA).
{\bf Essa into sim está perfeita :-)}

%***************************************************************% 
%                                                               %
%                              PCA                              %
%                                                               % 
%***************************************************************%

\section{Principal Component Analisys}
\label{sec:PCAeTomoPCA:PCA}

Baseada em encontrar os eixos com maiores variâncias em um conjunto de variáveis
(no nosso caso, fluxos por lambda e por zona espacial), a técnica de PCA vem sendo de grande
utilidade quando o assunto é estatística com muitas variáveis. Através de
operações relativamente simples usando álgebra linear, é
feita uma rotação na base de dados original, gerando uma nova base ortonormal
não correlacionada através de um conjunto de autovalores e autovetores.

Existem diversas formas de se calcular essa base final. A prova matemática que
você pode obter essa base é feita através de multiplicadores de Lagrange,
calculando os autovetores ($\mathbf{e}{}_k$) e autovalores  ($\lambda_k$) {\bf\ojo ?? Melhor usar $\Lambda_k$ pra nao confundir com comprimento de onda...]} que
maximizam o valor de $\mathbf{e}{}_k^T \cdot \mathbf{C}{}_{cov} \cdot
\mathbf{e}{}_k$ (ver Eq. \ref{eq:PCA:covMatrix}) sujeito à restrição de que um
autovetor deve ser ortogonal a qualquer outro da base ($\mathbf{e}{}_i^T
\mathbf{e}{}_j = 0$) e que todos devem ser normalizados ($\mathbf{e}{}_i{}^T
\mathbf{e}{}_i = 1$) \citep[][p. 5-6]{JolliffePCA1986}. No caso caso de PCA com
espectros, cada um pode ser represtado como um ponto num espaço
\ojo$N_\lambda$-dimensional. Vários espectros formam uma núvem de pontos nesse espaço. Assim,
encontramos quais são os eixos mais significativos desse espaço em relação à
variância, sujeitos às restrições acima. Fazemos esse cálculo encontrando os
autovetores e autovalores da matriz de correlação {\bf \ojo Covariancia?]} desse espaço (a matriz tem
dimensão $\lambda \time \lambda$) usando a biblioteca científica
\texttt{SciPy}\footnote{\url{http://scipy.org/}} (\ref{fig:PCA:covMatrix}), que
encontra os autovetores e autovalores da matriz de correlação {\bf \ojo Covariancia?]}.

\subsection{PCA das galáxias do CALIFA}

Conforme a Seção \ref{sec:CALePyC:PyCASSO} vimos que o cubo de espectros das
galáxias do CALIFA estão acessíveis no PyCASSO separados por zonas, muitas correspondendo a píxeis individuiais e outras a conjuntos de pixeis (zonas de Voronoi). Os espectros
observados estão armazenados em forma de uma matriz $(n \times m)$ 
{\bf [\ojo Conferir notacao .... $n = N_z$?? $m = N_\lambda$??]} 
com $n$ zonas
e $m$ comprimentos de onda (\texttt{f\_obs}\footnote{No PyCASSO está amostrado
de forma transposta ($\lambda \times z$) a usada.} no PyCASSO). 

\begin{equation}
    \label{eq:PCA:fluxMatrix}
    \textbf{F}{}_{z \lambda} = \left[
    \begin{array}{ccccc}
        f_{z_0 \lambda_0} & f_{z_0 \lambda_1} & f_{z_0 \lambda_2} & ... & f_{z_0 \lambda_m} \\
        f_{z_1 \lambda_0} & f_{z_1 \lambda_1} & f_{z_1 \lambda_2} & ... & f_{z_1 \lambda_m} \\
        f_{z_2 \lambda_0} & f_{z_2 \lambda_1} & f_{z_2 \lambda_2} & ... & f_{z_2 \lambda_m} \\
        ...               & ...               & ...               & ... & ...               \\
        f_{z_n \lambda_0} & f_{z_n \lambda_1} & f_{z_n \lambda_2} & ... & f_{z_n \lambda_m} 
    \end{array} 
    \right]
\end{equation}

Calculamos então o espectro médio de uma galáxia 

\begin{equation}
\langle
\textbf{F}{}_\lambda \rangle = \frac{1}{n} \sum_{i=0}^{n} f_{z_i}{}_{\lambda}
\end{equation}

\noindent e então subtraímos a média de todos os espectros

\begin{equation}
\textbf{I}{}_{z \lambda} =
\textbf{F}{}_{z \lambda} - \langle \textbf{F}{}_\lambda \rangle
\end{equation}

\noindent para o cálculo
da matriz de covariâncias usando um conjunto de dados com média zero:'

\begin{equation}
	\label{eq:PCA:covMatrix}
	\mathbf{C}{}_{cov} = \frac{[\mathbf{I}{}_{z \lambda}]^T \cdot \mathbf{I}{}_{z
	\lambda}}{n - 1}
\end{equation}

\noindent Vemos que
a matriz de covariância possui dimensão $\lambda \times \lambda$. 
{\bf[\ojo $m \times m$?? $N_\lambda \times N_\lambda$??]}

Agora calculamos os autovalores e autovetores da matriz de covariância. Neste
trabalho usamos o nome autoespectro para designar esses autovetores pois são de
uma matriz de covariâncias entre espectros de cada zona. Então ordenamos-os
decrescentemente pelo valor de seus autovalores. Os autoespectros são as componentes principais (PCs) e
os autovalores as respectivas variâncias. Isso feito, temos então o que necessitamos para
iniciar o cálculo do Tomograma PCA.

\begin{figure}
\begin{python}
# Carregar arquivo FITS com os dados.
from pycasso import fitsQ3DataCube
K = fitsQ3DataCube('K0277_synthesis_suffix.fits')

# Calcular o espectro medio de uma galaxia. 
# K.f_obs tem dimensao (lambda, zona), portanto, 
# fazemos o espectro medio de todas as zonas.
f_obs_mean__l = K.f_obs.mean(axis = 1)

# Subtraimos a media
I_obs__zl = K.f_obs.transpose() - f_obs_mean__l

# Calcular a matrix de convariancia
import scipy as sp
n = K.N_zone
dot_product = sp.dot(I_obs__zl.transpose(), I_obs__zl)
covMat__ll = dot_product / (n - 1.0)   

# Calcular os autovalores e autovetores
w, e = linalg.eigh(covMat__ll)

# Ordenar os autovetores decrescentemente pelo seu autovalor
S = sp.argsort(w)[::-1]
eigval = W[S]
eigvect = e[:, S]
 
\end{python}
	\caption[Exemplo de cálculo de PCA usando o PyCASSO e SciPy] {Cálculo do
	procedimento completo de PCA para os espectros observados de uma galáxia do
	CALIFA usando o PyCASSO e a biblioteca científica de Python chamada
	\texttt{SciPy}. No final do código temos em eigval e eigvect os
	autovalores e autovetores ordenados em forma decrescente.
	{\bf\ojo O TEXTO NAO FAZ REF A ESTA "FIG"!!}
	}
	\label{fig:PCA:covMatrix}
\end{figure}

Muitas figuras de PCs e suas utilizações e diferenças nos pré-processamentos
serão mostradas nos Capítulos \ref{sec:UsoPCA} e \ref{sec:result}, juntamente
com a Tomografia PCA e as comparações com os parâmetros físicos da sintese de
populações estelares com o \starlight.

%***************************************************************%
%                                                               %
%                        Tomografia PCA                         %
%                                                               %
%***************************************************************%

\section{Tomografia PCA}
\label{sec:PCAeTomoPCA:TomoPCA}

Na técnica de PCA procuramos os autoespectros (PCs) da matriz de correlação,
ordenados pela variância, formam uma base ($\mathbf{E}{}_{\lambda k}$) onde 
podemos projetar os nossos dados através da transformação:

\begin{equation}
	\label{eq:TomoPCA:tomogram2D}
	\mathbf{T}{}_{z k} = \mathbf{I}{}_{z \lambda} \cdot \mathbf{E}{}_{\lambda k}
\end{equation}

Projetamos nossa matriz de observáveis com a média subtraida ($\mathbf{I}{}_{z
\lambda}$) na base das PCs ($\mathbf{E}{}_{\lambda k}$). Na posse dessa nova
matriz transformada e de um mapa que leve de zona para uma par de coordenada ($z
\to (x, y)$), podemos montar assim uma imagem. Cada imagem funciona como uma
``fatia'' de um cubo de dados expandido na nova base, assim formando a
Tomografia PCA\footnote{\url{http://www.astro.iag.usp.br/~pcatomography/}},
criada e assim batizada por \citet{Steiner2009}, que em seu artigo faz um
paralelo com fatias de um espaço tridimensional (tomograma do corpo humano, por
exemplo) ou no espaço de velocidades (Tomografia Doppler). Cada ``fatia'' possui
um autoespectro relacionado que, em conjunto, trazem novas perspectivas e ideias
para a interpretação de ambos. 




\section{Exemplos de resultado da Tomagrafia PCA}

Para ilustrar o potencial da tecniva de tomog PCA, ilustramos os resultados obtidos em 2 artigos.

\subsection{Descoberta de linhas largas em um LINER}

No artigo citado anteriormente, através do estudo dos autoespectros e suas
respectivas imagens do núcleo da galáxia LINER ({\em Low Ionizations Nuclear
Emission-line Region}) NGC 4736, foram encontradas evidências de linhas largas. Quando temos uma fonte que é capaz de
produzir linhas largas no espectro é sinal da existência de um SMBH ({\em Super
Massive Black Hole}). \citet{CidFernandes2004} mostram que, pelo menos em alguns casos, a subtração 
detalhada das populações estelares nos espectros ajuda a encontrar linhas largas
mais fracas em Seyferts 2, que são aquelas que (por definição)  possuem apenas linhas estreitas,
ajudando assim na classificação desses objetos como tipo 1 ou 2. O PCA,
juntamente com a Tomografia PCA, fazem esse papel da subtração das populações
estelares sem haver nenhuma parametrização.

Steiner et al .... estudaram os 8 primeiros autoespectros, escolhidos através de um {\em scree
test} (\ref{fig:TomoPCA:scree}), no qual se verifica a variância de cada PC e
toma-se as mais relevantes. O autoespectro com mais variância (no artigo tratado
com E1) possui $99.74\%$ da variância e reproduz comportamento do gás e da
população estelar somados (Figura \ref{fig:TomoPCA:eigspec1}). O segundo
contribui com $0.088\%$ para a variância e tem um claro padrão de rotação, tanto
nas linhas do autoespectro quanto na imagem da tomografia (Figura
\ref{fig:TomoPCA:eigspec2}). É no terceiro ($0.032\%$ da variância) que
mostra evidência clara de uma emissão larga de \Halpha (Figura
\ref{fig:TomoPCA:eigspec3}). Essa assinatura é uma evidência tipica de AGNs de tipo 1.


\begin{figure}
    \includegraphics[width=0.7\textwidth]{figuras/figSteiner2009fig1.pdf}
    \caption[{\em Scree test} na galáxia NGC 4736.]
    {Scree test das primeiras 16 PCs do cubo de espectros da região
    central da galáxia NGC 4736. 
    Retirado de \citet[][fig. 1]{Steiner2009}.}
    \label{fig:TomoPCA:scree}
\end{figure}

\begin{figure}
    \includegraphics[width=0.9\textwidth]{figuras/figSteiner2009figA1.pdf}
    \caption[Tomograma e autoespectro 1 da galáxia NGC 4736.]
    {Autoespectro 1 e seu respectivo tomograma. Retirado de \citet[][fig.
    A1]{Steiner2009}.}
    \label{fig:TomoPCA:eigspec1}
\end{figure}

\begin{figure}
    \includegraphics[width=0.9\textwidth]{figuras/figSteiner2009figA2.pdf}
    \caption[Tomograma e autoespectro 2 da galáxia NGC 4736.]
    {Autoespectro 2 e seu respectivo tomograma. Retirado de \citet[][fig.
    A2]{Steiner2009}.}
    \label{fig:TomoPCA:eigspec2}
\end{figure}

\begin{figure}
    \includegraphics[width=0.9\textwidth]{figuras/figSteiner2009figA3.pdf}
    \caption[Tomograma e autoespectro 3 da galáxia NGC 4736.]
    {Autoespectro 2 e seu respectivo tomograma. Retirado de \citet[][fig.
    A3]{Steiner2009}.}
    \label{fig:TomoPCA:eigspec3}
\end{figure}


\subsection{Reflexao da luz de um AGN escondido em NGC 7097}

{\bf\ojo A FAZER!! \ojo Ricci et al 2011. Figs 3 e 5. vais precisar da minha ajuda para entender esse assunto de reflexao ... le o paper e depois falamos}




\section{Tomografia PCA para dados CALIFA: diferenças com respeito a trabalhos anteriores}

Os exemplos acima nos motivam a aplicar esta mesma técnica aos cubos de dados do CALIFA, e o resto desta dissertacao apresenta nossos expeimentos nesse sentido. Cumpre ressaltar, contudo, que o tipo de dados que analisaremos difere bastante dos dados analisados até agora com essa nova ferramenta. 


Essas diferenças são tanto de caracter observacional, como metodologico e físico. Para começo de conversa, Steiner e colaboradores trabalham com dados do Gemini (8m) com ?? min de integracao, enquanto nossos dados vêm de integracoes de tipicamente ?? min no telescópio de 3.5m do observatório de Calar Alto. 

Além disso, os trabalhos de ?? e ?? aplicam tecnicas de deconvolucao das imagens com o algoritmo de Richardson-Lucy. Dado o pequeno FoV do IFU do Gemini, o ganho com tais técnicas é perceptível. Nao aplicaremos isso no nosso caso pq nosso FoV é muito maior, e não esperamos grandes ganhos em resolucao espacial com a aplicacao dessa técnica de deconvolucao espacial.

Como dito na sec ???, a escala e resolucao espacial de nossos dados difere brutalmente daquelas nos trabalhos acima resumidos: 3 x 5 arcsec e resolucao de ?? arcsec no Gemini contra ?? x ?? e resolcao de 3.7 arcsec no CALIFA. Portanto, o FoV estudado por Steiner e colaboradores equivalen a aproximadamente 2 elementos de resolucao do CALIFA! Isto de cara mostra que a ciencia que podemos obter com Tomografia PCA + CALIFA  nao será a mesma que aquela estudade por Steiner et al.

Em resumo, apesar de inspirado diretamente pelos resultados de Steiner et al, este estudo difere muito dos trabalhos deles. Estas diferencas ficarao evidentes já a partir do próximo capítulo, onde apresentamod nossos primeiros resultados e discutimos algumas variacoes com respeito ao processamento de Steuer et al que decidimos fazer para melhor se adequar ao contexto de dados CALIFA.

%% End of this chapter
