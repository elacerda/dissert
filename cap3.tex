%%%%%%%%%%%%%%%%%%%%%%%%%%%%%%%%%%%%%%%%%%%%%%%%%%%%%%%%%%%%%%%%%
% Dissertacao de Mestrado / Dept Fisica, CFM, UFSC              %
% Lacerda@UFSC - 2013                                           %
%%%%%%%%%%%%%%%%%%%%%%%%%%%%%%%%%%%%%%%%%%%%%%%%%%%%%%%%%%%%%%%%%

%:::::::::::::::::::::::::::::::::::::::::::::::::::::::::::::::%
%                                                               %
%                          Capítulo 3                           %
%                                                               %
%:::::::::::::::::::::::::::::::::::::::::::::::::::::::::::::::%

%***************************************************************%
%                                                               %
%                      PCA e Tomografia PCA                     %
%                                                               %
%***************************************************************%

\chapter{PCA e Tomografia PCA}
\label{sec:PCAeTomoPCA}

De medidas fisiológicas, como pulsação e respiração, até reconhecimento de
padrões em sistemas complexos como reconhecimento facial e criptografia,
passando por compactação de imagem, neurosciência, redução de ruídos em sinal
gerado por transdures de qualquer espécie (óticos, acústicos, sísmicos), podemos ver
atuação de técnicas de PCA. \ojo Brincando com a própria técnica, o PCA em si é
uma CP de um universo matemático-estatístico.

% ***************************************************************% % PCA        
%                       % %
% ***************************************************************%

\section{Principal Component Analisys}
\label{sec:PCAeTomoPCA:PCA}

Baseada em encontrar os eixos com maiores variâncias em um conjunto de variáveis
(no nosso caso, fluxos por lambda e por zona), a técnica PCA vem sendo de grande
utilidade quando o assunto é estatística com muitas variáveis. Através de
operações relativamente simples computacionalmente usando álgebra linear, é
feito uma rotação na base de dados original, gerando uma nova base ortonormal
não correlacionada através de um conjunto de auto-valores e auto-vetores.

\fixme Existem diversas formas de se cálcular essa base final, mas a forma mais
formal de se obter os auto-valores e auto-vetores seria através do cálculo variacional,
usando restrições matemáticas, como o fato do auto-vetor ter que ser ortonormal
a qualquer outro da base e o auto-valor ser aquele que maximiza a variância
entre os auto-vetores \citep[][cap 1]{JolliffePCA1986}. 
\fixme covariâncias entre todos os seus vetores de dados.
Para isso, de maneira mais formal matemáticamente, fazemos uso de cálculo
variacional  Para esse cálculos usamos uma função pronta da biblioteca de
científica de Python chamada \texttt{SciPy}\footnote{\url{http://scipy.org/}}.

Conforme a Seção \ref{sec:CALePyC:PyCASSO} vimos que os dados das galáxias estão
acessíveis no PyCASSO separados por zonas. Então, por exemplo, os espectros
estão armazenados em forma de uma matriz $(n x m)$ com $n$ zonas e $m$
comprimentos de onda.

\begin{equation}
    \label{eq:PCA:fluxMatrix}
    \textbf{F}{}_{z \lambda} = \left[
    \begin{array}{ccccc}
        f_{z_0 \lambda_0} & f_{z_0 \lambda_1} & f_{z_0 \lambda_2} & ... & f_{z_0 \lambda_m} \\
        f_{z_1 \lambda_0} & f_{z_1 \lambda_1} & f_{z_1 \lambda_2} & ... & f_{z_1 \lambda_m} \\
        f_{z_2 \lambda_0} & f_{z_2 \lambda_1} & f_{z_2 \lambda_2} & ... & f_{z_2 \lambda_m} \\
        ...               & ...               & ...               & ... & ...               \\
        f_{z_n \lambda_0} & f_{z_n \lambda_1} & f_{z_n \lambda_2} & ... & f_{z_n \lambda_m} 
    \end{array} 
    \right]
\end{equation}

Calculamos então o espectro médio de uma galáxia através da equação $\langle
\textbf{F}{}_\lambda \rangle = (1 / n) \sum_{i=0}^{n} f_{z_i}{}_{\lambda}$ e
então subtraímos a média de todos os espectros ($\textbf{I}{}_{z \lambda} =
\textbf{F}{}_{z \lambda} - \langle \textbf{F}{}_\lambda \rangle$) para o cálculo
da matriz de covariâncias usando um conjunto de dados com média zero.

\begin{equation}
	\label{eq:PCA:covMatrix}
	\mathbf{C}{}_{cov} = \frac{[\mathbf{I}{}_{z \lambda}]^T \cdot \mathbf{I}{}_{z
	\lambda}}{n - 1}
\end{equation}

\begin{figure}
\begin{python}
# Carregar arquivo FITS com os dados.
from pycasso import fitsQ3DataCube
K = fitsQ3DataCube('K0277_synthesis_suffix.fits')

# Calcular o espectro medio de uma galaxia. 
# K.f_obs tem dimensao (lambda, zona), portanto, 
# fazemos o espectro medio de todas as zonas.
f_obs_mean__l = K.f_obs.mean(axis = 1)

# Subtraimos a media
I_obs__zl = K.f_obs.transpose() - f_obs_mean__l

# Calcular a matrix de convariancia
import scipy as sp
n = K.N_zone
covMat__ll = (1.0 / n) * sp.dot(I_obs__zl.transpose(), I_obs__zl) 

\end{python}
	\caption[Exemplo de cálculo da matriz de covariância usando o PyCASSO]
	{Cálculo da matriz de covariância usando o PyCASSO e a biblioteca científica
	de Python chamada \texttt{SciPy}}
	\label{fig:PCA:covMatrix}
\end{figure}

Na Figura \ref{fig:PCA:covMatrix} podemos ver a implementação de um código
Python usando o PyCASSO e a biblioteca científica \texttt{SciPy}.

%***************************************************************%
%                                                               %
%                        Tomografia PCA                         %
%                                                               %
%***************************************************************%

\section{Tomografia PCA}
\label{sec:PCAeTomoPCA:TomoPCA}
\ojo
Aqui vão as fórmulas, exemplos e referências da Tomografia PCA.

%% End of this chapter
