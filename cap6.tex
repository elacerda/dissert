%%%%%%%%%%%%%%%%%%%%%%%%%%%%%%%%%%%%%%%%%%%%%%%%%%%%%%%%%%%%%%%%%
% Dissertacao de Mestrado / Dept Fisica, CFM, UFSC              %
% Lacerda@UFSC - 2013                                           %
%%%%%%%%%%%%%%%%%%%%%%%%%%%%%%%%%%%%%%%%%%%%%%%%%%%%%%%%%%%%%%%%%

%:::::::::::::::::::::::::::::::::::::::::::::::::::::::::::::::%
%                                                               %
%                          Capítulo 6                           %
%                                                               %
%:::::::::::::::::::::::::::::::::::::::::::::::::::::::::::::::%

%***************************************************************%
%                                                               %
%                           Conclusao                           %
%                                                               %
%***************************************************************%

\chapter{Conclusões e perspectivas}
\label{sec:conclusao}

%***************************************************************%
%                                                               %
%                    Conclusao - este trabalho                  %
%                                                               %
%***************************************************************%

\section{Este trabalho}

O objetivo principal deste trabalho foi explorar os cubos de espectros das galáxias do {\em survey} CALIFA utilizando a
técnica de Tomografia PCA. Talvez pela primeira vez em IFS de galáxias inteiras, mas com certeza pela primeira vez com
os dados deste {\em survey}, que cobrem $\sim100\%$ da luz de uma galáxia, reamostrada em spaxels de 1 arcsec$^2$,
formando um cubo de espectros $\mathbf{F}_{x,y,\lambda}$.

A descoberta de linhas largas (elementos típicos de AGNs tipo 1) na galáxia NGC 4736 classificada como LINER, através
da técnica de tomografia PCA desenvolvida no trabalho de S09, usando dados de IFS do Gemini da região central da galáxia
($\sim100$ pc) serviram de base teórica para nosso trabalho.

Desenvolvemos um programa, por ora apelidado de ``PCAlifa'', utilizando linguagem Python e os {\em pipelines} {\sc
qbick} e PyCASSO, ambos construidos pela colaboração entre o projeto CALIFA e o GAS-UFSC, que serviu de base para todo
nosso estudo exploratório. Estudo este sobre 4 galáxias espirais: NGC 0001 (CALIFA 8), NGC 0776 (CALIFA 73), NGC 2916
CALIFA 277) e NGC 4210 (CALIFA 518); 2 galáxias {\em early-type}: NGC 1167 (CALIFA 119) e NGC 6515 (CALIFA 864); 2 {\em
mergers}: NGC 2623 (CALIFA 213) e ARP 220 (CALIFA 802). 

Encontramos diversas correlações entre as primeiras PCs (autoespectros) e propriedades físicas de populações estelares
obtidos através da síntese para os cubos de espectros das galáxias do CALIFA, feito por CF13 utilizando o \starlight.
Essas correlações formam uma certa engenharia reversa no sentido de buscar os parâmetros que mais ``importantes'' (em
variância) dos dados e as propriedades físicas, de uma forma não-paramétrica. Por fim, especulamos que o parâmetro
$\Lambda$ pode ser uma medida de complexidade do sistema estudado dando uma ideia do tamanho da núvem de pontos no
espaço $N_\lambda$ dimensional.

\section{Trabalhos futuros}

{\bf\ojo ISSO VAI PARA O CAP 6 / PERSPECTIVAS / FUTURE-WORK: 

(Algumas das cousas a fazer)

Aplicar a cubos da versão 1.4 da pipeline de reducao do CALIFA. Esta nova versão ficou pronta há poucos meses, e apenas
em Janeiro foi testada por nossos colaboradores. Pode-se perceber que esta nova reducao produz espectros melhor
calibrados. Por exemplo, os resíduos espectrais dos ajustes com o \starlight\ mostrados na fig ??? de CF13 para a v1.3
ficaram bem melhores com a 1.4.

Em todo esse estudo trabalhamos com os cubos rearranjados em zonas de Voronoi. Essa escolha se baseia única e
exclusivamente em nosso desejo de comparar as PCs com os resultados do \starlight, que requer um $S/N$ para ter
resultados fiáveis. Porém, esse esquema acaba por deteriorar muito a qualidade das imagens para zonas externas, que
ficam muito gordas... e perdemos resolucao espacial. Seria interessante fazer o PCA com o cubo original, não zonificado.
Isto tvz produza tomogramas que revelem melhor as estruturas como braços espirais e regioes HII nas partes externas das
galáxias.

Outras técnicas como ICA, wavelets, e etc poderiam tvz trazer melhorias a nossa análise. Ricci et al ??? usam algumas
dessas técnicas, mas nós sequer tocamos nisso. É uma possível linha de trabalho para o futuro.
}
\subsection{Linhas de emissão e intervalos específicos em comprimento de onda}
\label{sec:UsoPCA:PCAlidades:emlin}

Podemos também executar o PCA apenas em intervalos específicos do espectro. Isso pode ser feito de várias formas, por
exemplo utilizando todos os pontos do(s) intervalo(s) em $\lambda$, ou utilizando apenas o fluxo integrado, ou apenas
larguras equivalentes de linhas, ou FWHM das linhas. Um exemplo aplicado é fazer o PCA apenas das regiões que abrangem
$\oIII/\Hbeta$ em conjunto com $\nII/\Halpha$ ou então $\mathrm{H}\delta$ e D$4000$ para estudar a variância espacial
desses ratios.


\ojo
Remover cinemática (v0 \& vd) pois é um desperdício de variância.
Incluir linhas de emissão na análise (aqui posso incluir umas figuras 
com essa parte que já está programada.

% End of this chapter
