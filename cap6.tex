%%%%%%%%%%%%%%%%%%%%%%%%%%%%%%%%%%%%%%%%%%%%%%%%%%%%%%%%%%%%%%%%%
% Dissertacao de Mestrado / Dept Fisica, CFM, UFSC              %
% Lacerda@UFSC - 2013                                           %
%%%%%%%%%%%%%%%%%%%%%%%%%%%%%%%%%%%%%%%%%%%%%%%%%%%%%%%%%%%%%%%%%

%:::::::::::::::::::::::::::::::::::::::::::::::::::::::::::::::%
%                                                               %
%                          Capítulo 6                           %
%                                                               %
%:::::::::::::::::::::::::::::::::::::::::::::::::::::::::::::::%

%***************************************************************%
%                                                               %
%                           Conclusao                           %
%                                                               %
%***************************************************************%

\chapter{Conclusões e perspectivas}
\label{sec:conclusao}

%***************************************************************%
%                                                               %
%                    Conclusao - este trabalho                  %
%                                                               %
%***************************************************************%

\section{Este trabalho}
\ojo
%***************************************************************%
%                                                               %
%                  Conclusao - trabalhos futuros                %
%                                                               %
%***************************************************************%

\section{Trabalhos futuros}

\subsection{Linhas de emissão e intervalos específicos em comprimento de onda}
\label{sec:UsoPCA:PCAlidades:emlin}

Podemos também executar o PCA apenas em intervalos específicos do espectro. Isso pode ser feito de várias formas, por
exemplo utilizando todos os pontos do(s) intervalo(s) em $\lambda$, ou utilizando apenas o fluxo integrado, ou apenas
larguras equivalentes de linhas, ou FWHM das linhas. Um exemplo aplicado é fazer o PCA apenas das regiões que abrangem
$\oIII/\Hbeta$ em conjunto com $\nII/\Halpha$ ou então $\mathrm{H}\delta$ e D$4000$ para estudar a variância espacial
desses ratios.


\ojo
Remover cinemática (v0 \& vd) pois é um desperdício de variância.
Incluir linhas de emissão na análise (aqui posso incluir umas figuras 
com essa parte que já está programada.

% End of this chapter
