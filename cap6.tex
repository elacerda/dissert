%%%%%%%%%%%%%%%%%%%%%%%%%%%%%%%%%%%%%%%%%%%%%%%%%%%%%%%%%%%%%%%%%
% Dissertacao de Mestrado / Dept Fisica, CFM, UFSC              %
% Lacerda@UFSC - 2013                                           %
%%%%%%%%%%%%%%%%%%%%%%%%%%%%%%%%%%%%%%%%%%%%%%%%%%%%%%%%%%%%%%%%%

%:::::::::::::::::::::::::::::::::::::::::::::::::::::::::::::::%
%                                                               %
%                          Capítulo 6                           %
%                                                               %
%:::::::::::::::::::::::::::::::::::::::::::::::::::::::::::::::%

%***************************************************************%
%                                                               %
%                           Conclusao                           %
%                                                               %
%***************************************************************%

\chapter{Conclusões e perspectivas}
\label{sec:conclusao}

%***************************************************************%
%                                                               %
%                    Conclusao - este trabalho                  %
%                                                               %
%***************************************************************%

\section{Este trabalho}
\label{sec:conclusao:thisWork}

O objetivo principal deste trabalho foi explorar os cubos de espectros das galáxias do {\em survey} CALIFA utilizando a
técnica de Tomografia PCA. Talvez pela primeira vez em IFS de galáxias inteiras, mas com certeza pela primeira com os
dados deste {\em survey}, que cobrem $\sim100\%$ da luz de uma galáxia, reamostrada em spaxels de 1 arcsec$^2$, formando
um cubo de espectros $\mathbf{F}_{x,y,\lambda}^{orig}$. 

Ao longo dos últimos anos a PCA vem se tornando uma técnica quase obrigatória para o estudo de conjuntos multivariados
de dados. Através dessa redução de dimensionalidade pressupõe-se que a busca pelas características mais importantes sejam
mapeadas pelas componentes principais. A descoberta de linhas largas (elementos típicos de AGNs tipo 1) na galáxia NGC
4736 classificada como LINER, através da técnica de tomografia PCA desenvolvida no trabalho de S09, usando dados de IFS
do Gemini da região central da galáxia ($\sim100$ pc) corrobora essas pressuposições, servindo de base teórica para
nosso trabalho. Cabe aqui ressaltar que o {\em FoV} coberto pelos dados do Gemini usados no trabalho de S09 equivale a
$\sim2$ elementos de resolução do CALIFA. Com um campo cobrindo toda galáxia, nossos dados mapeiam regiões com
estruturas muito diferentes (e.g., bojo, disco, regiões H{\sc ii}).

Desenvolvemos um programa, por ora apelidado de ``PCAlifa'', utilizando linguagem Python e os {\em pipelines} {\sc
qbick} e \pycasso, ambos construidos pela colaboração entre o projeto CALIFA e o GAS-UFSC, que serviu de base para todo
nosso estudo exploratório. Estudo este sobre 4 galáxias espirais: NGC 0001 (CALIFA 8), NGC 0776 (CALIFA 73), NGC 2916
CALIFA 277) e NGC 4210 (CALIFA 518); 2 galáxias {\em early-type}: NGC 1167 (CALIFA 119) e NGC 6515 (CALIFA 864); 2 {\em
mergers}: NGC 2623 (CALIFA 213) e ARP 220 (CALIFA 802). Em duas das galáxias escolhidas pudemos, através da Tomografia
PCA, identificar problemas com os espectros que passaram para o cubo final mesmo com os tratamentos dos {\em pipelines}
de redução e os filtros de qualidade do {\sc qbick}.

Encontramos diversas correlações entre as primeiras PCs (autoespectros) e propriedades físicas de populações estelares
obtidos através da síntese para os cubos de espectros das galáxias do CALIFA, feito por CF13 utilizando o \starlight.
Essas correlações formam uma certa engenharia reversa no sentido de buscar os parâmetros mais ``importantes'' em
variância (as PCs) que mapeiam propriedades físicas, de uma forma não-paramétrica.

\section{Trabalhos futuros}
\label{sec:conclusao:futWorks}

Cumprimos um papel exploratório inicial utilizando os dados de IFS do CALIFA. Temos diversas vertentes surgindo a partir
deste trabalho, mas a primeira delas é poder fazer a mesma análise utilizando a nova versão do {\em pipeline} de redução
do CALIFA (v1.4). Essa nova versão ficou pronta há poucos meses e apenas em Janeiro foi testada por nossos
colaboradores. Através desses testes preliminares já pôde-se perceber que esta nova redução produz espectros com melhor
calibração.

Em todo esse estudo trabalhamos com os cubos rearranjados em zonas de Voronoi. Essa escolha se baseia única e
exclusivamente em nosso desejo de comparar as PCs com os resultados do \starlight, que requer um $S/N$ grande o
suficiente para obter resultados fiáveis. Porém, esse agrupamento acaba por deteriorar muito a qualidade das imagens
para zonas externas, que, quanto mais afastadas do núcleo, possuem mais pixels agrupados em uma zona, perdendo assim
resolução espacial. Seria interessante fazer o PCA com o cubo original, não zonificado. Isto talvez produza tomogramas
que revelem melhor as estruturas como braços espirais e regioes H{\sc ii} nas partes externas das galáxias.

Um outra possível linha de trabalho é poder explorar as qualidades de filtro da PCA, seja para reconstruir os espectros
originais selecionando determinadas características encontradas em diferentes PCs ou para suavização dos espectros
através da redução do ruído causado por todo processo observacional, de redução e de criação do COMBO. De posse da
matriz de autoespectros ($\mathbf{E}_{\lambda k}$) podemos reconstruir a matriz $\mathbf{I}_{z \lambda}$  calculada pela
equação \ref{eq:PCA:Izl} utilizando $r \leq k$ autoespectros, onde $k$ é o número total de autoespectros, através da
transformação
\begin{equation}
	\mathbf{I}{}_{z \lambda}^{rec}(r \leq k) = \mathbf{T}{}_{z k}(r \leq k) \cdot [\mathbf{E}{}_{\lambda k}(r \leq k)]^T
\end{equation}
\noindent assim nos permitindo reconstruir o cubo de espectros ($\mathbf{F}_{z \lambda}^{rec}$) adicionando o
espectro médio $\langle \mathbf{F}{}_\lambda \rangle$ a $\mathbf{I}{}_{z \lambda}^{rec}$:
\begin{equation}
	\mathbf{F}{}_{z \lambda}^{rec} = \mathbf{I}{}_{z \lambda}^{rec} + \mean{\mathbf{F}_\lambda}
\end{equation}
 \noindent Por exemplo, S09 reconstroem o espectro do AGN da galáxia NGC 4736 selecionado determinadas componentes.
 Outro exemplo é a reconstrução dos espectros com a supressão de componentes que caracterizem cinemática (PCs que
 correlacionam com $v_\star$ e $\sigma_\star$). Outras técnicas podem ser aliadas ao PCA, como, por exemplo, faz
 \citet{Riffel2011} utilizando Tomografia PCA juntamente com uma técnica de {\em wavelet} com o intuito de filtragem do
 cubo de espectros.

Podemos também executar o PCA apenas em intervalos específicos do espectro. Isso pode ser feito de várias formas, por
exemplo utilizando todos os pontos do(s) intervalo(s) em $\lambda$, ou utilizando apenas o fluxo integrado, ou apenas
larguras equivalentes de linhas. Um exemplo aplicado é fazer o PCA apenas das regiões que abrangem $\oIII/\Hbeta$ em
conjunto com $\nII/\Halpha$ ou então $\mathrm{H}\delta$ e D4000 para estudar a variância espacial desses ratios.

Um outro experimento que podemos realizar é fazer a PCA utilizando como entrada o logarítmo dos fluxos observados. Essa
variante da análise pode ser vista como uma mera curiosidade matemática, uma experiência para ver o que se obtém. No
entanto, ela é inspirada em um motivo físico: é razoável esperar que utilizando o logarítmo possamos ressaltar efeitos
devidos a gradientes de poeira, já que a extinção atua de forma multiplicativa sobre os espectros.

Devido a característica de {\em legacy-survey} do CALIFA e do fato que cada galáxia é uma amostra estatística
{\em per se}, muitas novidades ainda virão através da ciência realizada com estes dados pelo grupo de pesquisadores do
IAA e também através da parceria com nossos colaboradores aqui do GAS-UFSC.

% End of this chapter
