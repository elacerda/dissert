%%%%%%%%%%%%%%%%%%%%%%%%%%%%%%%%%%%%%%%%%%%%%%%%%%%%%%%%%%%%%%%%%
% Dissertacao de Mestrado / Dept Fisica, CFM, UFSC              %
% Lacerda@UFSC - 2013                                           %
%%%%%%%%%%%%%%%%%%%%%%%%%%%%%%%%%%%%%%%%%%%%%%%%%%%%%%%%%%%%%%%%%

%:::::::::::::::::::::::::::::::::::::::::::::::::::::::::::::::%
%                                                               %
%                        Agradecimentos                         %
%                                                               %
%:::::::::::::::::::::::::::::::::::::::::::::::::::::::::::::::%

\chapter*{Agradecimentos}

Difícil reunir agradecimentos para todas as contribuições de tanta gente boa que conheci durante esses últimos dois
anos. Fácil mesmo é saber que sem minha família nada disso seria possível. Mãe e avós (Leila, Lourdes e Neide), sem
vocês nada haveria. Seria impossível. Meus irmãos, Victória e José, sempre presentes. Tios e primos também sempre me
dando força. Mesmo longe, estavam perto. Um agradecimento em especial ao meu pai {\em in memorian}, obrigado por me
apontar a direção!

Quando, há três anos atrás, o meu grande amigo Prof. Kahio T. Mazon me indicou para meu orientador, Prof. Roberto Cid
Fernandes Jr., eu não sabia o quanto ia ser legal. Aliás, tanto o Cid quanto o Kahio foram verdadeiros pais durante
esses últimos anos. Obrigado a vocês dois que sempre souberam quando e como me ensinar e nunca deixaram proporcionar
momento que me incentivem a seguir nessa caminhada!

Falando em ensinar e incentivar, agradeço também ao casal de amigos e professores Sônia e Fred, pelas horas de papo em
corredores e salas de aulas da UFSC, me ajudando a enxergar as dificuldades e belezas das carreiras de professor e
pesquisador.

Ao Prof. João Steiner e seus colaboradores pelo trabalho seminal desenvolvido. Ao Prof. Fabrício Ferrari pela sua
participação na etapa inicial (e agora na final também como membro da banca examinadora). Esse trabalho também não seria
possível sem o projeto CALIFA e todos os seus colaboradores. Muito obrigado a vocês. Em especial a Prof. Rosa M.
Gonzalez Delgado e aos Profs. Sebastian F. Sánchez e Enrique Pérez, que são parte basilar para o andamento do projeto.
Agradeço também ao amigo e pesquisador Rubén García Benito.

Durante esses dois anos pude viajar por muitos lugares e não passei em nenhum deles sem deixar amigos. Obrigado a todos
aqueles que conheci através da astrofísica, seja pelas reuniões da SAB, ou pelas viagens em congressos, workshops
e encontros, no Rio Grande do Sul, em São Paulo, na Venezuela ou no México. Em especial a Prof. Gra\.zyna Stasi\'nska e ao
Prof. Christophe Morisset - um dos melhores fotógrafos que já conheci - que já se tornaram mais do que amigos. Obrigado
também aos amigos Alfredo Mejía, Glória Delgado e Ivan Cabrera! Estão sempre convidados a voltar ao Brasil.

Aos amigos da universidade. Em especial aos professores do GAS-UFSC: Abílio Mateus, Antônio Kanaan, Natalia Asari e
Raymundo Baptista. Os amigos astrofísicos André, William, Bernardo, Marielli, Herpich, Germano, Rafael e Ariel; meus
amigos e irmãos de casa, Maykot e Hari, além dos que nos acompanham nas insalubridades da vida, mesmo que por um
curto, mas intenso, tempo: André 1 e 2, Bruno, Suzane e Ubiratã, Jéssica, Roger, Mateus ``pinguim'', Adriano ``PCM'',
Diego e Vanessa e todos aqueles que de alguma forma me apoiaram durante este tempo. Quero agradecer também aos
professores e funcionários do Programa de Pós-Graduação em Física – especialmente ao Chefe de Expediente do PPGFSC,
Antônio Machado.

Não poderia deixar de aqui agradecer aos amigos que estou em falta (e que também fazem muita falta). Acredito que minha
vida será sempre melhor com a compreensão, amizade e amor de vocês, meus compadres, Fernando e Vivian, os amigos da
música Leonardo e Adriana, Márcio e Elisabeth, Felipe e Sabrina, e ao meu irmão escolhido, Marcus Vinicius. Obrigado
também a todas as suas famílias que ao longo dos anos já são parte da minha.

Obrigado a todos vocês!

% End of Acknowledgments
