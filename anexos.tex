%%%%%%%%%%%%%%%%%%%%%%%%%%%%%%%%%%%%%%%%%%%%%%%%%%%%%%%%%%%%%%%%%
% Dissertacao de Mestrado / Dept Fisica, CFM, UFSC              %
% Andre@UFSC - 2011                                             %
%%%%%%%%%%%%%%%%%%%%%%%%%%%%%%%%%%%%%%%%%%%%%%%%%%%%%%%%%%%%%%%%%


%:::::::::::::::::::::::::::::::::::::::::::::::::::::::::::::::%
%                                                               %
%                            Anexos                             %
%                                                               %
%:::::::::::::::::::::::::::::::::::::::::::::::::::::::::::::::%

%***************************************************************%
%                                                               %
%                    Manual Starlight + Galex                   %
%                                                               %
%***************************************************************%

\chapter{{\em Queries} SQL}
\label{apendice:Queries}

{\em Queries} SQL utilizadas no texto. As palavras-chave da linguagem
estão em negrito e maiúsculas. Os servidores {\em CasJobs} referenciados na
legenda das figuras estão listados abaixo.

\begin{list}{}{\setlength\itemsep{0pt}}
\item \starlight: \url{http://casjobs.starlight.ufsc.br/casjobs/}
\item \SDSS: \url{http://casjobs.sdss.org/CasJobs/}
\item \galex: \url{http://galex.stsci.edu/casjobs/}
\end{list}



\begin{figure}
\begin{Verbatim}[frame=single,commandchars=\\\{\}]
\textbf{UPDATE} sample
	\textbf{SET} SpecObjID=so.SpecObjID, ObjID=so.BestObjID
\textbf{FROM} sample s2 \textbf{INNER JOIN} DR7..SpecObjAll so
	\textbf{ON} so.MJD=s2.MJD
	\textbf{AND} so.Plate=s2.Plate
	\textbf{AND} so.FiberID=s2.FiberID
\end{Verbatim}
	\caption
	[{\em Query} para atualizar os índices da amostra de galáxias do
	\starlight.]
	{Atualização dos índices da amostra de galáxias do \starlight. A {\em query}
	foi executada no {\em CasJobs} do \SDSS para obter {\tt SpecObjID} e {\tt
	BestObjID} dado o tripleto [{\tt MJD}, {\tt Plate}, {\tt FiberID}].}
	\label{fig:QueryAtualizaObjIds}
\end{figure}


\begin{figure}
\begin{Verbatim}[frame=single,commandchars=\\\{\}]
\textbf{SELECT}
	'AIS' \textbf{AS} survey, '1' \textbf{AS} mg, '1' \textbf{AS} ms, \textbf{COUNT(*)} \textbf{AS} N
\textbf{FROM} xsdssdr7 x
\textbf{INNER JOIN} photoobjall p \textbf{ON} x.objid = p.objid 
\textbf{INNER JOIN} photoextract pe \textbf{ON} p.photoextractid = pe.photoextractid
\textbf{WHERE} x.multipleMatchCount = 1 \textbf{AND} x.reverseMultipleMatchCount = 1
	\textbf{AND} pe.mpstype='AIS'

\textbf{UNION SELECT}
	'AIS' \textbf{AS} survey, '1' \textbf{AS} mg, '2' \textbf{AS} ms, \textbf{COUNT(*)} \textbf{AS} N
\textbf{FROM} xsdssdr7 x
\textbf{INNER JOIN} photoobjall p \textbf{ON} x.objid = p.objid 
\textbf{INNER JOIN} photoextract pe \textbf{ON} p.photoextractid = pe.photoextractid
\textbf{WHERE} x.multipleMatchCount = 1 \textbf{AND} x.reverseMultipleMatchCount = 2
	\textbf{AND} pe.mpstype='AIS'

\textbf{UNION SELECT}
	'AIS' \textbf{AS} survey, 1 \textbf{AS} mg, '+' \textbf{AS} ms, \textbf{COUNT(*)} \textbf{AS} N
\textbf{FROM} xsdssdr7 x
\textbf{INNER JOIN} photoobjall p \textbf{ON} x.objid = p.objid 
\textbf{INNER JOIN} photoextract pe \textbf{ON} p.photoextractid = pe.photoextractid
\textbf{WHERE} x.multipleMatchCount = 1 \textbf{AND} x.reverseMultipleMatchCount > 2
	\textbf{AND} pe.mpstype='AIS'

\textbf{UNION SELECT}
\textbf{...}
\end{Verbatim}
	\caption[{\em Query} listando identificações entre \SDSS e {\em surveys} do
	\galex.] {Lista das identificações mútuas entre \SDSS DR7 e os {\em surveys}
	do \galex. São contados quantos objetos de um dado {\em survey} têm
	identificação direta e reversa com apenas 1, 2 ou mais de objetos. Os campos
	\texttt{mg} e \texttt{ms} representam respectivamente o número de candidatos
	para o \galex e para o \SDSS. Ver seção
	\ref{sec:Crossmatch:DefAmostras:IdSDSSGalex} para mais detalhes sobre a
	identificação cruzada. Apenas uma parte da {\em query} foi incluída, o
	restante é similar ao código presente. {\em Query} executada no {\em CasJobs}
	do \galex.}
	\label{fig:QueryIDGalexSDSS}
\end{figure}


\begin{figure}
\begin{Verbatim}[frame=single,commandchars=\\\{\}]
\textbf{SELECT INTO} mydb..galex_ais
	s.objid \textbf{AS} sdssobjid, x.objid \textbf{AS} galexobjid,
	s.mjd, s.plate, s.fiberid,
	g.fuv_mag, fuv_magErr,
	g.nuv_mag, g.nuv_magErr,
	g.e_bv,
	g.band,
	x.distance,
	pe.fexptime,
	pe.nexptime
\textbf{FROM} mydb..sample s
\textbf{LEFT JOIN} xSDSSDR7 x
	\textbf{ON} s.objid = x.sdssobjid
	\textbf{AND} x.distanceRank=1
	\textbf{AND} x.reverseDistanceRank=1
	\textbf{AND} x.multipleMatchCount=1
	\textbf{AND} x.reverseMultipleMatchCount=1
\textbf{LEFT JOIN} photoobjall g
	\textbf{ON} g.objid = x.objid
\textbf{LEFT JOIN} photoextract e
	\textbf{ON} e.photoextractid=g.photoextractid
\textbf{WHERE} e.mpstype='AIS'
\end{Verbatim}
	\caption[{\em Match} entre os objetos da amostra do \starlight e \galex.]
	{{\em Query} para o {\em match} entre os objetos da amostra do \starlight e
	\galex AIS. A mesma {\em query} foi usada para o MIS, trocando apenas o nome da
	tabela para {\tt galex\_mis} e modificando a última linha para {\tt
	e.mpstype='MIS'}. {\em Query} executada no {\em CasJobs} do \galex.}
	\label{fig:QueryMatchAIS}
\end{figure}


\begin{figure}
\begin{Verbatim}[frame=single,commandchars=\\\{\}]
\textbf{SELECT}
	'AIS' \textbf{AS} survey, 'FUV' \textbf{AS} band, \textbf{COUNT(*)} \textbf{AS} N
\textbf{FROM} galex_ais
\textbf{WHERE} galexobjid <> 0
	\textbf{AND} fuv_mag <> -999
\textbf{UNION SELECT}
	'AIS' \textbf{AS} survey, 'NUV' \textbf{AS} band, \textbf{COUNT(*)} \textbf{AS} N
\textbf{FROM} galex_ais
\textbf{WHERE} galexobjid <> 0
	\textbf{AND} nuv_mag <> -999
\textbf{UNION SELECT}
	'AIS' \textbf{AS} survey, 'FUV+NUV' \textbf{AS} band, \textbf{COUNT(*)} \textbf{AS} N
\textbf{FROM} galex_ais
\textbf{WHERE} galexobjid <> 0
	\textbf{AND} fuv_mag <> -999
	\textbf{AND} nuv_mag <> -999
	
\textbf{UNION SELECT}
	'MIS' \textbf{AS} survey, 'FUV' \textbf{AS} band, \textbf{COUNT(*)} \textbf{AS} N
\textbf{FROM} galex_mis
\textbf{WHERE} galexobjid <> 0
	\textbf{AND} fuv_mag <> -999
\textbf{UNION SELECT}
	'MIS' \textbf{AS} survey, 'NUV' \textbf{AS} band, \textbf{COUNT(*)} \textbf{AS} N
\textbf{FROM} galex_mis
\textbf{WHERE} galexobjid <> 0
	\textbf{AND} uv_mag <> -999
\textbf{UNION SELECT}
	'MIS' \textbf{AS} survey, 'FUV+NUV' \textbf{AS} band, \textbf{COUNT(*)} \textbf{AS} N
\textbf{FROM} galex_mis
\textbf{WHERE} galexobjid <> 0
	\textbf{AND} fuv_mag <> -999
	\textbf{AND} nuv_mag <> -999
\end{Verbatim}
	\caption
	[{\em Query} para listar as detecções por banda UV e {\em survey}.]
	{Lista contendo quantidade de objetos no catálogo \starlight+UV com deteções
	\galex nas bandas FUV, NUV e em ambas. Valores de $-999$ indicam que o a
	coluna está indefinida. Valores de \texttt{galexObjID} iguais a zero
	indicam que o objeto \starlight não tem correspondente {\galex}. {\em Query}
	executada no {\em CasJobs} do \starlight.}
	\label{fig:QueryListaPorBandaSurvey}
\end{figure}


\begin{figure}
\begin{Verbatim}[frame=single,commandchars=\\\{\}]
\textbf{SELECT INTO} MyDB..galex_ais_elines_z
	s.specobjid,
	g.nuv_mag \textbf{AS} NUV,
	o.Mu, o.Mg, o.Mr, o.Mi, o.Mz,
	o.m_u, o.m_g, o.m_r, o.m_i, o.m_z,
	s.mcor_gal, s.at_flux, s.at_mass, s.am_flux,
	s.am_mass, s.AV,
	e.oiii_5007_flux, e.oiii_5007_flux_err,
	e.oiii_5007_ew, e.oiii_5007_ew_err, e.oiii_5007_sn,
	e.nii_6584_flux, e.nii_6584_flux_err,
	e.nii_6584_ew, e.nii_6584_ew_err, e.nii_6584_sn,
	e.halpha_flux, e.halpha_flux_err,
	e.halpha_ew, e.halpha_ew_err, e.halpha_sn,
	e.hbeta_flux, e.hbeta_flux_err,
	e.hbeta_ew, e.hbeta_ew_err, e.hbeta_sn,
	o.z \textbf{AS} redshift
\textbf{FROM} galex_ais g
\textbf{INNER JOIN} synthesis_results s \textbf{ON}
	s.specobjid = g.specobjid
\textbf{INNER JOIN} el_fit_all e \textbf{ON}
	s.synid = e.synid
\textbf{INNER JOIN} obs_parameters o \textbf{ON}
	o.specobjid = s.specobjid
\textbf{WHERE}
	g.galexobjid <> 0
	\textbf{AND} g.nuv_mag <> -999
	\textbf{AND} o.z > 0.04
	\textbf{AND} o.z < 0.17
	\textbf{AND} o.m_r < 17.77
\end{Verbatim}
	\caption[Extração da amostra \starlightUV.]
	{Extração da amostra \starlightUV. Os limites em {\em redshift}
	($0,04 < z < 0,17$) e magnitude ($r<17,77$) são explicados na seção
	\ref{sec:Crossmatch:DefAmostras:StarlightUV}. O critério ``\texttt{g.nuv\_mag
	<> -999}'' remove os objetos que não tiveram identificação em NUV. {\em Query}
	executada no {\em CasJobs} do \starlight.}
	\label{fig:QuerySampleAIS}
\end{figure}


% End of this chapter
