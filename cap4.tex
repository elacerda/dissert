%%%%%%%%%%%%%%%%%%%%%%%%%%%%%%%%%%%%%%%%%%%%%%%%%%%%%%%%%%%%%%%%%
% Dissertacao de Mestrado / Dept Fisica, CFM, UFSC              %
% Lacerda@UFSC - 2013                                           %
%%%%%%%%%%%%%%%%%%%%%%%%%%%%%%%%%%%%%%%%%%%%%%%%%%%%%%%%%%%%%%%%%


%:::::::::::::::::::::::::::::::::::::::::::::::::::::::::::::::%
%                                                               %
%                          Capítulo 4                           %
%                                                               %
%:::::::::::::::::::::::::::::::::::::::::::::::::::::::::::::::%

%***************************************************************%
%                                                               %
%                       Testes com PCA                          %
%                                                               %
%***************************************************************%

\chapter{PCA nos espectros}
\label{sec:UsoPCA}

O uso de métodos estatísticos já se estende por séculos em praticamente todas (senão todas) as áreas de conhecimento.
Esse fato cria uma necessidade de que existam cada vez mais estudos sobre estudos, ou {\em metaestudos}\footnote{Em
alusão a metadados, que são dados sobre dados.}. Precisamos saber de que maneira os pré-processamentos de nossa amostra
afetam os dados e, principalmente, o resultado após a aplicação de determinada técnica, para que dessa forma o
desenvolvimento não se torne uma ``caixa preta'' inacessível.

\section{Pré-processamento dos cubos}
\label{sec:UsoPCA:PCAlidades}

Antes dos espectros chegarem ao PyCASSO, todas as informações de {\em flags}\footnote{Marcações.} em {\em bad pixels} e
linhas telúricas\footnote{Linhas de emissão ou absorção referentes à atmosfera.} são criadas em um {\em pipeline} de
pré-processamento chamado {\sc qbick}. Esse {\em pipeline} também prepara os cubos para a execução do \starlight, que
posteriormente serão organizados pelo PyCASSO, definindo as zonas de Voronoi, a reamostragem em $\lambda$ e colocando os
espectros em repouso usando o {\em redshift} calculado dentro dos $5"$ centrais da galáxia. Todas essas informações e
pré-processamentos provenientes do {\sc qbick} são herdadas pelo PyCASSO e já estão contidadas em seus cubos de
espectros.

Apesar desses pré-processamentos supracitados, não é sobre eles vamos falar aqui, e sim sobre aqueles que são feitos nos
espectros contidos no PyCASSO antes da aplicação do PCA. Como o PCA é uma técnica em qual calcula-se os eixos que,
através da variância, melhor expandem sua base de dados, é natural que qualquer pré-processamento que altere a variância
dos dados, resultará num conjunto diferente de PCs. Quando aplicamos o PCA aos cubos do CALIFA estamos buscando
variâncias espaciais nos espectros. Durante nossas investigações fizemos uma série de testes com pré-processamentos nos
espectros. Dois deles são constantes em todos os estudos. Primeiro todos os espectros são limitados ao intervalo de
$3800$ a $6850$ \AA. Após essa limitação, fazemos uma estatística com todos os {\em bad pixels} e linhas telúricas de
cada cubo e removemos qualquer um que esteja presente em mais de $5\%$ dos espectros. Cabe aqui lembrar que todos os
espectros precisam ter os mesmos pontos em $\lambda$ pois precisamos construir a matriz de covariância. Podemos ver o
efeito desses pré-processamentos nos espectros através dos primeiro e segundo espectros na Figura
\ref{fig:UsoPCA:checkmask}.

\begin{figure}
    %\includegraphics[height=0.5\textwidth]{figuras/figHusemann2013Fig2.pdf}
    \includegraphics[width=1.0\textwidth]{figuras/K0277-constant_inital_mask-399.pdf}
    \caption[Exemplo de máscaras em um espectro do cubo de dados.]
    {Espectro da zona 399 da galáxia NGC 2916 (CALIFA 277). Acima está o espectro completo. No segundo vemos o espectro
    com linhas telúricas e bad-pixels removidos, além do limite de intervalo em comprimendo de onda de $3800$ a $6850$
    \AA. No espectro mais abaixo, além das partes removidas no segundo, estão foram removidas também as mesmas linhas de
    emissão mascaradas na síntese de populações estelares.}
    \label{fig:UsoPCA:checkmask}
\end{figure}

\begin{figure}
    \includegraphics[width=1.\textwidth]{figuras/K0277-fobs_norm.pdf}
    \caption[Fluxos de normalização para cada zona da galáxia K0277.]
    {A imagem à esquerda e a superior direita, mostram o fluxo usado para a normalização de cada espectro em
    cada pixel (à esquerda) ou zona (superior direita). Na imagem inferior direita temos um histograma para valores do
    fluxo de normalização.}
    \label{fig:UsoPCA:K277fobsnorm}
\end{figure}

\subsection{Normalização}

Como exemplo inicial apresentamos as quatro primeiras PCs e seus respectivos tomogramas do cubo de espectros observados
da galáxia NGC 2916 (Figura \ref{fig:UsoPCA:K277tomofobs}). O tomograma é formado pelo ``peso'' da PC em cada pixel (no
nosso caso zona). A primeira PC é praticamente o espectro médio, que fica evidente comparando com o espectro médio em
cinza no segundo eixo vertial (à direita). O segundo autoespectro se assemelha com um espectro de população muito jovem.
Com o auxilio de seu tomgorama vemos que para as zonas mais internas da galáxia (\fixme onde se presume que as
populações sejam mais velhas \citneed) o seu peso é negativo. De forma contrária, em regiões mais afastadas do núcleo,
sobre os braços espirais, onde, para essa galáxia, parecem existir algumas regiões HII, vemos que seu peso é positivo. A
terceira componente mostra um padrão de rotação, ao mesmo tempo misturado com um fator de escala, falaremos mais dela no
fim dessa seção. A quarta componente se assemelha a média também, mas fica mais complicado dizer o que ela representa.
No final desse capítulo e no próximo vamos falar mais sobre como melhor interpretar as PCs fisicamente.

Da mesma forma, fazemos o PCA para o mesmo cubo de dados, mas normalizados da forma que comentamos em \fixme ({\it
\textcolor{red}{Preciso saber aonde colocar um texto sobre essa parte de como é feita a normalização dos espectros.
Acredito que vá para o Cap. 2 mas vou esperar pra juntar com o que tu fizeste nos cap 1, 2 e 3}}) (Figura
\ref{fig:UsoPCA:K277tomofobsnorm}). Podemos ver que se assemelha muito ao PCA feito com os dados sem normalização
deslocando uma componente acima, ou seja, o mesmo PCA mas sem a primeira componente. A PC de maior variância no caso sem
normalização não adiciona informação alguma a análise das populações, funciona como um fator de escala. O espectro médio
já é a informação necessária que precisaríamos, portanto a primeira componente para o caso sem normalização não trás
informação adicional. Veremos a seguir que seu tomograma também não trás maiores informações.

Imagine uma galáxia composta inteiramente pela mesma população estelar, em repouso, distribuídas da mesma maneira no
espaço. Ou seja, em qualquer ponto da galáxia o espectro é o mesmo. Um PCA nessa galáxia hipotética nos mostraria apenas
uma componente relevante. Adicione então a essa galáxia uma função de densidade de massa em função do raio, permitindo
que de uma posição para outra a quantidade dessa determinada população se altera (mudando o brilho superficial da
região). Uma componente nova irá surgir na sua análise PCA, mostrando que existe uma variância agora numa componente de
escala (amplitude) nos espectros. Mas o que essa componente de escala nos diz sobre a física da população estelar
existente? Essa componente seria um disperdício de variância para uma análise de populações estelares de uma galáxia.

No caso do CALIFA, com um {\em FoV} abrangendo praticamente toda a galáxia, esse efeito de amplitude se acentua muito
devido ao brilho superficial \ojo mais intenso nas zonas centrais da galáxia em comparação com as mais afastadas, assim
adicionando uma grande variância descartável entres as zonas. Descartável pois não trazem informação nova para a nossa
análise. Essas diferenças em amplitude não nos dizem nada sobre as populações estelares. Comparemos agora a primeira
PC do caso sem normalização (Figuras \ref{fig:UsoPCA:K277tomofobs}) e a imagem mais à esquerda na Figura
\ref{fig:UsoPCA:K277fobsnorm} formada pelos fluxo para normalização por zona. Podemos notar que o primeiro autoespectro
(e seu respectivo tomograma) no caso sem normalização mostra exatamente esse fator de escala. Seu tomograma mostra que
ela é claramente um fator de escala (que pode ser considerado um fator de brilho, ou de amplitude, nos espectros). Por
esse motivo, em nossas análises daqui para frente, usaremos o cubo com os espectros normalizados.

\begin{figure}
    %\includegraphics[height=0.5\textwidth]{figuras/figHusemann2013Fig2.pdf}
    \includegraphics[width=1.\textwidth]{figuras/K0277-tomo1a4.pdf}
    \caption[Tomogramas de 1 a 4 da gal\'axia NGC 2916 - $F_{obs}$.]
    {Quatro primeiros PCs (e seus respectivos tomogramas) do PCA aplicado aos espectros sem normalização da galáxia
    NGC 2916.}
    \label{fig:UsoPCA:K277tomofobs}
\end{figure}
\begin{figure}
    %\includegraphics[height=0.5\textwidth]{figuras/figHusemann2013Fig2.pdf}
    \includegraphics[width=1.\textwidth]{figuras/K0277-tomo1a4-norm.pdf}
    \caption[Tomogramas de 1 a 4 da gal\'axia NGC 2916 - $F_{obs} / F_{\lambda 5365}$.]
    {Quatro primeiros PCs (e seus respectivos tomogramas) do PCA aplicado aos espectros com normalização da galáxia
    NGC 2916.}
    \label{fig:UsoPCA:K277tomofobsnorm}
\end{figure}

\subsection{Cinemática}

Usando novamente a ideia da galáxia hipotética com apenas uma população estelar, imagine agora que elas estão
distribuídas uniformemente, mas estão em rotação com a galáxia. Da mesma forma o espectro de todas será igual salvo por
deslocamentos em $\lambda$ no espectro. Esses efeitos cinemáticos não estão nos trazendo informação alguma para o estudo
das populações estelares. Causam um grande disperdício de variâncias, sempre aparecendo nas primeiras PCs. Também é
importante lembrar que existem métodos mais eficazes e direcionados para a determinação de tais propriedades
cinemáticas. Nas galáxias presentes no CALIFA não poderia ser diferente, portanto os espectros aparecem com linhas
deslocadas para o azul ({\em blue-shifted}) ou para o vermelho ({\em red-shifted}) dependendo da velocidade de rotação
projetada. A disperção de velocidades em cada ponto da galáxia também pode causar alargamento ou estreitamento das
linhas.

Podemos ver um padrão bem claro de rotação na PC2 da Figura \ref{fig:UsoPCA:K277tomofobsnorm}. Da mesma forma, na PC3 da
Figura \ref{fig:UsoPCA:K277tomofobs}. Nesta última, fizemos alguns ajustes na saturação das cores de modo que diminuisse
o efeito do fator de escala ainda presente, de modo que ficasse mais evidente o padrão de rotação. É claro que como não
há a normalização, ou seja, ainda há uma mistura de variância pelas intensidades misturado nessa PC, o padrão de rotação
não é tão evidente, diferentemente da PC2 do PCA com normalização.

\section{Comparando as PCs com o \STARLIGHT: engenharia reversa}
\label{sec:UsoPCA:EngRev}

Nos espectros, além dos {\em bad pixels} e linhas telúricas, podemos mascarar regiões desnecessárias para determinada
investigação científica. Nosso foco é o estudo das populações estelares, portanto necessitamos que as linhas de emissão,
geralmente associadas ao gás presente nas galáxias \fixme sejam removidas do espectro. Para que possamos fazer
correlações entre os resultados do PCA e as propriedades físicas obtidas pela síntese através do \starlight, em nossos
espectros também são mascaradas todas as regiões do espectro que não entram na síntese ($\mathrm{H}\epsilon$: de $3960$
a $3980$ \AA; $\mathrm{H}\delta$: de $4092$ a $4112$ \AA; $\mathrm{H}\gamma$: de $4330$ a $4350$ \AA; \Hbeta: de $4848$
a $4874$ \AA; \oIII: de $4940$ a $5028$ \AA; $\mathrm{He\,\textsc{i}}$ e $\mathrm{NaD}$: de $5866$ a $5916$ \AA; \Halpha
e \nII: de $6528$ a $6608$ \AA; $\mathrm{S\,\textsc{ii}}$: de $6696$ a $6752$ \AA). O espectro após todas máscaras pode
ser visto no terceiro espectro (o mais abaixo) da Figura \ref{fig:UsoPCA:checkmask}.

\subsection{Fluxos observados e sintéticos}

Com o resultado da síntese de populações estelares já organizado para as galáxias do CALIFA, realizamos o PCA no cubo de
espectros observados e no de espectros sintéticos. A grande diferença é que nos espectros sintéticos estão contidas
apenas as informações sobre populações estelares\footnote{Suavização, correções por poeira e cinemática também são
feitas nos espectros no processo de síntese. Mais detalhes em \citet{CidFernandes2005}}. Como os espectros sintéticos
não possuem as assinaturas dos equipamentos observacionais, de todo o processo de redução e afins, quando analisadas
pela técnica de PCA vemos que as informações se condensam em menos PCs. Comparando os dois {/em scree tests} na Figura
\ref{fig:UsoPCA:K0277scree} vemos que para o caso com os espectros sintéticos a curva converge mais rápido ao zero de
variância, mostrando que temos as informações mais compactadas nas primeiras PCs quando comparadas ao caso com os
espectros observados. Observando o caso sem normalização, plotado no gráfico com linha pontilhada, vemos que o efeito
causado pelo fator de escala (PC1) diminui a ``importância'' das demais PCs. As quatro primeiras PCs e seus tomogramas
provenientes do cubo com os fluxos sintéticos normalizados da galáxia NGC 2916, aparece na Figura
\ref{fig:UsoPCA:K277tomofsynnorm}. Vemos que ela se assemelha muito àqueles da Figura \ref{fig:UsoPCA:K277tomofobsnorm}.

\begin{figure}
    \includegraphics[width=1.\textwidth]{figuras/K0277-screetest.pdf}
    \caption[Scree test comparativo entre 3 PCAs.]
    {Scree test para 3 análises PCA da galáxia NGC 2916 (CALIFA 277). Com marcações de triângulos vemos as PCs
    resultantes do PCA com os espectros observados normalizados. As variâncias das PCs marcadas com estrela representam
    o PCA com os espectros sintéticos normalizados. Para comparação plotamos as PCs do caso sem normalização usando
    linha pontilhada.}
    \label{fig:UsoPCA:K0277scree}
\end{figure}


\begin{figure}
    %\includegraphics[height=0.5\textwidth]{figuras/figHusemann2013Fig2.pdf}
    \includegraphics[width=1.\textwidth]{figuras/K0277-tomo1a4-syn-norm.pdf}
    \caption[Tomogramas de 1 a 4 da gal\'axia NGC 2916 - $F_{syn} / F_{\lambda 5365}$ .]
    {Quatro primeiros PCs (e seus respectivos tomogramas) do PCA aplicado aos espectros sintéticos normalizados da
    galáxia NGC 2916.}
    \label{fig:UsoPCA:K277tomofsynnorm}
\end{figure}

\subsection{Correla\c{c}\~oes}

Caso as grandezas físicas analisadas em uma galáxia fossem não-correlacionadas seria muito mais fácil fazer uma
comparação entre PCs e grandezas físicas, mas geralmente não é o que ocorre. Para colhermos informações sobre os objetos
no céu só temos duas formas: através imagens ou de espectros. Em ambas formas, diferentes efeitos físicos causam efeitos
semelhantes nas cores (imagem) ou nos espectros. Esses efeitos fazem com que alguns dos parâmetros físicos fiquem
extremamente correlacionados.\ojo \citneed \textcolor{red}{Gostaria de adicionar aqui algumas referências como alguns
dos papers do SEAGal/Starlight que fala sobre as degenerescencias de idade e metalicidade e algumas outras grandezas
correlacionadas. Coloquei essa parte aqui pensando naquela crítica que o cara me fez lá no México sobre as grandezas em
astrofísica serem extremamente correlacionadas, mas não sei até que ponto é legal comentar sobre isso visto que
estamos analisando as coisas via correlação\ldots talvez essa discussão possa ficar mais pro Cap 6 nas conclusões.}.

\ldots \dots \ldots \ldots

\textcolor{red}{DAQUI PRA BAIXO AINDA É O ESQUELETO}
\ojo A Filosofia desse capítulo é aprender/testar como operar o PCA para que ele
reflita isso ou aquilo... descrevemos uma série de experimentos nesse sentido.

Preprocessamentos e diferentes tipos de PCAs com ou sem linhas, diferentes
faixas espectrais, com dados normalizados ou não. (importante!!!)

Vamos nos limitar a no-emission lines analysis, descrever a máscara de linhas de
emissão, etc. Isso para facilitar a coisa e pq queremos correlar o resultado do
PCA com os dados do Starlight (PyCASSO).

Simulações para ajudar a decifrar os resultador, população jovem + velha +
modelo de distribuição espacial - ver efeitos de estratégias de
preprocessamento.

Correlacionando os resultados do PCA com as propriedades do starlight (tipo de
engenharia reversa)

Linhas telúricas - remover ou não... bad pixels... mascarar ou não linhas de
emissão