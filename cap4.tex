%%%%%%%%%%%%%%%%%%%%%%%%%%%%%%%%%%%%%%%%%%%%%%%%%%%%%%%%%%%%%%%%%
% Dissertacao de Mestrado / Dept Fisica, CFM, UFSC              %
% Lacerda@UFSC - 2013                                           %
%%%%%%%%%%%%%%%%%%%%%%%%%%%%%%%%%%%%%%%%%%%%%%%%%%%%%%%%%%%%%%%%%


%:::::::::::::::::::::::::::::::::::::::::::::::::::::::::::::::%
%                                                               %
%                          Capítulo 4                           %
%                                                               %
%:::::::::::::::::::::::::::::::::::::::::::::::::::::::::::::::%

%***************************************************************%
%                                                               %
%                       Testes com PCA                          %
%                                                               %
%***************************************************************%

\chapter{Uso do PCA}
\label{sec:UsoPCA}

O uso de métodos estatísticos já se estende por séculos em praticamente todas
(senão todas) as áreas de conhecimento. Esse fato cria uma necessidade de que
existam cada vez mais estudos sobre estudos, ou {\em metaestudos}\footnote{Em
alusão a metadados, que são dados sobre dados.}. Precisamos saber de que maneira
os pré-processamentos de nossa amostra afetam os dados e, principalmente, o
resultado após a aplicação de determinada técnica, para que dessa
forma, o desenvolvimento não se torne uma ``caixa preta'' inacessível.

\section{Pré-processamento dos cubos}
\label{sec:UsoPCA:PCAlidades}

Durante a construção deste trabalho, podemos ver que cada mudança no
pré-processamento no cubo de dados original gerava resultados diferente na
técnica de PCA e, consequentemente, na Tomografia PCA. Iclusive fizemos
comparações usando o cubo de espectros observados e o de espectros modelados pelo \starlight
\citep{CidFernandes2013I}.

Como o PCA é uma técnica em qual calcula-se os eixos que, através da variância,
melhor expandem sua base de dados, é natural que qualquer pré-processamento que
altere a variância dos dados resultará num conjunto diferente de PCs.
Modificações de caráter físico também podem ser feitas nos espectros, como, por
exemplo, a correção de cinemática ou a remoção de linhas espectrais de emissão
(ou absorção). Um outro exemplo de pré-processamento apenas de específicos
intervalos na SED, procurando variâncias em linhas espectrais ou regiões de
interesse com diferentes linhas ou {\em line-ratios}. O importante é,
sabendo que isso ocorre, tentar entender e manipular as consequências de cada
pre-processamento.

 saber que a cada
pré-processamento nos dados pode alterar (algumas vezes muito) os resultados.





\ojo A Filosofia desse capítulo é aprender/testar
como operar o PCA para que ele reflita isso ou aquilo... descrevemos uma série
de experimentos nesse sentido.

Preprocessamentos e diferentes tipos de PCAs com ou sem linhas, diferentes 
faixas espectrais, com dados normalizados ou não. (importante!!!)

Vamos nos limitar a no-emission lines analysis, descrever a máscara de linhas 
de emissão, etc. Isso para facilitar a coisa e pq queremos correlar o 
resultado do PCA com os dados do Starlight (PyCASSO).

Simulações para ajudar a decifrar os resultador, população jovem + velha + 
modelo de distribuição espacial - ver efeitos de estratégias de 
preprocessamento.

Correlacionando os resultados do PCA com as propriedades do starlight (tipo de 
engenharia reversa)

Linhas telúricas - remover ou não... bad pixels... mascarar ou não linhas de 
emissão