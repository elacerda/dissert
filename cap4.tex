%%%%%%%%%%%%%%%%%%%%%%%%%%%%%%%%%%%%%%%%%%%%%%%%%%%%%%%%%%%%%%%%%
% Dissertacao de Mestrado / Dept Fisica, CFM, UFSC              %
% Lacerda@UFSC - 2013                                           %
%%%%%%%%%%%%%%%%%%%%%%%%%%%%%%%%%%%%%%%%%%%%%%%%%%%%%%%%%%%%%%%%%


%:::::::::::::::::::::::::::::::::::::::::::::::::::::::::::::::%
%                                                               %
%                          Capítulo 4                           %
%                                                               %
%:::::::::::::::::::::::::::::::::::::::::::::::::::::::::::::::%

%***************************************************************%
%                                                               %
%                       Testes com PCA                          %
%                                                               %
%***************************************************************%

\chapter{Aplicação a cubos de dados CALIFA}
\label{sec:PCAaplic}


Com os dados e ferramentas apresentados nos capítulos anteriores estamos
prontos para aplicar a Tomografia PCA a galáxias do CALIFA. Como o PCA é uma técnica que calcula os eixos que, através da variância,
melhor expandem a base de dados, é natural que qualquer pré-processamento que modifique os espectros de entrada acarrete mudancas na variância entre eles e portanto resulte num conjunto diferente de PCs. 
Durante nossas investigações fizemos uma série de testes com pré-processamentos e
manipulações nos espectros (normalização, remoção de cinemática, janelas diferentes em comprimento de onda, PCA das
linhas de emissão, logarítmo do fluxo, entre outros), verificando suas implicações no resultado. Neste Capítulo
comentaremos alguns {\em insights} que tivemos nesse processo, e definiremos operacoes que julgamos adequadas a analise PCA de galáxias inteiras. 
Por fim faremos uma espécie de engenharia reversa, 
estudando correlacoes entra PCs e propriedades derivadas da síntese de populacoes estelares, o que nos auxilia na busca do sentido
físico das PCs resultantes. 

Neste Capítulo todas as análises serão feitas usando a galáxia NGC 2916 (CALIFA 277), uma espiral de tipo Sbc. 
A discussão detalhada desse caso particular serve de guia para os resultados para outras galaxias de nossa amostra, apresentados
no
Capítulo \ref{sec:result}.

\section{PCA dos espectros observados: Efeitos de amplitude}
\label{sec:PCAaplic:norm}

Como exemplo inicial apresentamos as cinco primeiras PCs e seus respectivos tomogramas do cubo de espectros observados
da galáxia NGC 2916. Imagens SDSS e CALIFA dessa galáxia espiral podem ser vistas na Fig.\ \ref{fig:Fig1DeCF13}, enquanto a 
a Fig.\ \ref{fig:PCAaplic:K277ExampleSpectra} mostra alguns dos espectros CALIFA desse objeto. 


\begin{figure}
    \caption[\ojo!]
    {Alguns espectros do cubo de dados de NGC 2916. Os painéis à esquerda mostram os espectros originais (em erg/s/cm2/A), enquanto os painéis à direita mostra os mesmos espectros normalizados pelo fluxo mediano na janela de $5635 \pm 45$ \AA. {\bf\ojo EDUARDO: escolher uns 5 a 10 espectros para esta fig!}}
    \label{fig:PCAaplic:K277ExampleSpectra}
\end{figure}


A Figura \ref{fig:PCAaplic:K277tomofobs}) apresenta os resultados da tomografia PCA para esse cubo de dados. O tomograma é formado pelo ``peso'' da PC em cada zona. O primeiro autoespectro
lembra o espectro médio (linha em cinza claro nos painéis à direita), embora muito mais azul.
 O segundo autoespectro se assemelha com um espectro de população muito jovem. Com o auxilio de seu
tomograma vemos que para as zonas mais internas da galáxia (onde as populações sao mais velhas, como vimos nas Figs ??2.5 e 2.7??) o seu peso é negativo. De forma contrária, em regiões mais afastadas do núcleo, 
onde, para essa galáxia, parecem existir algumas regiões HII, vemos que seu peso é positivo. A terceira componente
parece mostrar um padrão de rotação, ao mesmo tempo misturada com um fator de escala. Para que esse padrão de rotação
ficasse visível tivemos que fazer um certo tratamento no tomograma, mas falaremos mais sobre isso no final dessa seção.
Da quarta componente em diante fica mais complicado dizer o que cada uma representa. No final desse capítulo falaremos mais sobre como melhor interpretar as PCs fisicamente.

Da mesma forma, fazemos o PCA para o mesmo cubo, mas com os espectros normalizados da forma que comentamos em \fixme
(Capítulo 2). Exemplos de espectros normalizados são mostrados nos painel direito da Fig.\ \ref{fig:PCAaplic:K277ExampleSpectra}.
A tomografia PCA desse cubo normalizado é mostrada na
Figura \ref{fig:PCAaplic:K277tomofobsnorm}. À primeira vista os resultados podem ser parecidos com o anterior
deslocando uma componente acima, ou seja, o mesmo PCA mas sem a primeira PC. Mas essa é só uma primeira impressão.  As
propriedades físicas de uma galáxia geralmente possuem simetria radial ou axial, sendo assim extremamente
correlacionadas no espaço. Matematicamente, a maior fonte de variância nos espectros é sua amplitude, aparecendo na
forma de componente principal como visto na análise feita sem normalização. Essa componente de escala, quando não suprimida,
correlaciona-se com as propriedades físicas da galáxia, portanto, deixa sua assinatura através de todas as PCs. Essa
amplitude, que gera a primeira PC no caso sem normalização não adiciona nenhuma informação qualitativa à análise das populações estelares,
funcionando apenas como um fator de escala, afetando as componentes seguintes. Esses vestígios de amplitude aparecem de
forma mais evidente quando comparamos a PC3 no caso sem normalização com a PC2 no caso com normalização. Podemos ver um
padrão de rotação na PC2 da Figura \ref{fig:PCAaplic:K277tomofobsnorm}. Com uma saturação na escala de cores no tomograma da
PC3 da Figura \ref{fig:PCAaplic:K277tomofobs} diminuímos o efeito do fator de escala ainda presente nessa componente, de
modo que ficasse mais evidente o padrão de rotação. O espectro médio já é a informação necessária que precisaríamos para
entender esse fator de escala nas amplitudes dos espectros, portanto a primeira componente para o caso sem normalização
não trás informação adicional. Veremos a seguir que seu tomograma também não trás maiores informações.

Imagine uma galáxia composta inteiramente pela mesma população estelar, em repouso, distribuídas da mesma maneira no
espaço. Ou seja, em qualquer ponto da galáxia o espectro é o mesmo. Um PCA nessa galáxia hipotética nao identificaria nenhuma fonte de variancia, pois todos espectros sao iguais ao espectro médio. Adicione então a essa galáxia uma função de densidade de massa estelar em função do raio (ou, equivalentemente, um perfil de brilho superficial), permitindo
que de uma posição para outra a quantidade dessa determinada população se altera (mudando a intensidade da
região). Uma componente nova irá surgir na sua análise PCA, mostrando que existe uma variância agora numa componente de
escala (amplitude) nos espectros. Mas o que essa componente de escala nos diz sobre as propriedades da população estelar
existente? Nada! Essa componente seria um desperdício de variância para uma análise de populações estelares de uma galáxia.

No caso do CALIFA, com um campo abrangendo praticamente toda a galáxia, esse efeito de amplitude se acentua muito
devido ao brilho superficial mais intenso nas zonas centrais da galáxia em comparação com as mais afastadas, assim
adicionando uma grande variância descartável entres as zonas. Descartável pois não trazem informação nova para a nossa
análise. Essas diferenças em amplitude não nos dizem nada sobre as populações estelares. Comparemos agora a primeira PC
do caso sem normalização (Figuras \ref{fig:PCAaplic:K277tomofobs}) e a imagem mais à esquerda na Figura
\ref{fig:PCAaplic:K277fobsnorm} formada pelos fluxo para normalização por zona. Podemos notar que o primeiro autoespectro
(e seu respectivo tomograma) no caso sem normalização mostra exatamente esse fator de escala. Seu tomograma mostra que
ela é claramente um fator de escala (que pode ser considerado um fator de brilho, ou de amplitude, nos espectros). 

Em suma, concluímos que é bem mais útil analisar espectros normalizados. 
Os trabalhos do grupo de J. Steiner não utilizaram esse esquema de normalizacao, mas isso é muito provavelmente devido ao pequeno campo coberto por seus dados. Nesse caso, as variacoes em amplitude sao pequenas e nao jogam um papel importante na analise. No caso do CALIFA, contudo, os efeitos de amplitude sao muito maiores, e complicam a interpretacao dos resultados da PCA. Por
esse motivo, em nossas análises daqui para frente usaremos o cubo com os espectros normalizados.

\begin{figure}
    \includegraphics[width=0.9\textwidth]{figuras/K0277-tomo-obs.pdf}
    \caption[Tomogramas de 1 a 5 da gal\'axia NGC 2916 - $F_{obs}$.]
    {Cinco primeiras PCs (e seus respectivos tomogramas) resultantes da Tomografia PCA aplicado aos espectros sem
    normalização da galáxia NGC 2916.}
    \label{fig:PCAaplic:K277tomofobs}
\end{figure}
\begin{figure}
    \includegraphics[width=0.9\textwidth]{figuras/K0277-tomo-obs-norm.pdf}
    \caption[Tomogramas de 1 a 5 da gal\'axia NGC 2916 - $F_{obs} / F_{\lambda 5365}$.]
    {Cinco primeiras PCs (e seus respectivos tomogramas) resultantes da Tomografia PCA aplicado aos espectros com
    normalização da galáxia NGC 2916.}
    \label{fig:PCAaplic:K277tomofobsnorm}
\end{figure}

\begin{figure}
    \includegraphics[width=1.\textwidth]{figuras/K0277-fobs_norm.pdf}
    \caption[Fluxos de normalização para cada zona da galáxia K0277.]
    {Fluxo usado para a normalização de cada espectro, mostrado tanto na forma de imagem (à esquerda) como em funcao do numero da zona (direita).}
    \label{fig:PCAaplic:K277fobsnorm}
\end{figure}

\section{PCA dos espectros sintéticos}
\label{sec:PCAaplic:OBSxSYN}


Na secao anterior 
realizamos o PCA no cubo de
espectros observados de CALIFA 277.  Com o resultado da síntese de populações estelares já organizado com o pipeline PyCASSO podemos aplicar a mesma tecnica aos espectros sintéticos gerados pelo \starlight.

A grande diferença é que nos espectros sintéticos estão contidas
apenas as informações sobre populações estelares\footnote{Efeitos de extinção/avermelhamento e cinemática também são
incluidos no processo de síntese (\citep{CidFernandes2005}}. Como os espectros sintéticos
não possuem as assinaturas dos equipamentos observacionais, dos processos para subtração do céu, ruídos e afins, quando
analisadas pela técnica de PCA espetamos que as informação se condense em menos PCs. Comparando os dois {\em scree tests}
na Figura \ref{fig:PCAaplic:K0277scree} vemos que para o caso com os espectros sintéticos a curva converge mais rápidamente ao
zero de variância, mostrando que, como esperado, temos as informações mais compactadas nas primeiras PCs quando comparadas ao caso com
os espectros observados. Observando o caso sem normalização, plotado no gráfico com linha pontilhada, vemos que o efeito
causado pelo fator de escala (PC1) diminui a ``importância'' (i.e., variância) das demais PCs. 

\begin{figure}
    \includegraphics[width=1.\textwidth]{figuras/K0277-screetest.pdf}
    \caption[Scree test comparativo entre 3 PCAs.]
    {Scree test para 3 análises PCA da galáxia NGC 2916 (CALIFA 277). Com marcações de triângulos vemos as PCs
    resultantes do PCA com os espectros observados normalizados. As variâncias das PCs marcadas com estrela representam
    o PCA com os espectros sintéticos normalizados. Para comparação plotamos as PCs do caso sem normalização usando
    linha pontilhada.}
    \label{fig:PCAaplic:K0277scree}
\end{figure}


As cinco primeiras PCs e
seus tomogramas provenientes do cubo com os fluxos sintéticos normalizados da galáxia NGC 2916 aparecem na Figura
\ref{fig:PCAaplic:K277tomofsynnorm}.

Comparando com seu correspondente obsetvacional na Fig.\ \ref{fig:PCAaplic:K277tomofobsnorm}, vemos que os resultados para os espectros sintéticos parecem ser mais ``limpos'', pois não existem
ruídos. Como são espectros gerados a partir de uma base teórica para diferentes idades e metalicidades de
populações estelares,  não possuem nenhuma assinatura instrumental. Por esse fato acabamos descobrindo uma
assinatura presente em quase todas as componentes geradas pelo PCA usando os dados observados. Como comentamos no
Capítulo \ref{sec:CALePyC:Apresent} utilizamos o cubo de dados COMBO, gerado a partir da união do V500 com o V1200. Como
os espectros V1200 possuem espectros com maior resolução do que os do V500 (V1200 - FWHM $\sim 2.3$ \AA; V500 - FWHM
$\sim 6$ \AA), o processo de criação do COMBO deixa vestígios. Para essa galáxia, a junção entre os dois cubos (V500 e
V1200) para a formação do COMBO acontece exatamente nesse intervalo de comprimento de onda ($\sim 4550$ \AA). O quinto autoespectro da
Figura \ref{fig:PCAaplic:K277tomofobsnorm} mostra um degrau entre $4500$ e $4600$ \AA o qual parece mostrar essa
diferença de comportamento entre as duas versões originais antes da formação do COMBO.  Já nas componentes sintéticas\footnote{Componentes geradas pelo PCA nos cubos de
espectros sintéticos.} não se vê essa mudança de comportamento no autoespectro.


Assim como na análise dos espectros observados normalizados, os efeitos da cinemática se fazem notar já na segunda PC dos cubos sinteticos, indicando que sao fonte impotante de variancia. Consideramos, porém, que essa é uma variância "descartável". Usando novamente a ideia da galáxia hipotética com
apenas uma população estelar, imagine agora que elas estão distribuídas uniformemente, mas estão em rotação com a
galáxia. Como anteriormente, o espectro de todos os píxels será igual, mas agora terá deslocamentos em $\lambda$. Esses
efeitos cinemáticos não estão nos trazendo informação alguma para o estudo das populações estelares. Causam um grande
desperdício de variâncias, sempre aparecendo nas primeiras PCs. Uma manipulação através do vetor de população\footnote{O
vetor de populações diz o quanto de cada populacao estelar da base entra na receita para construir o espectro sintético.} criado
pela síntese, juntamente com os espectros da base, pode nos ajudar a criar os espectros sintéticos sem nenhuma correção
por cinemática e poeira de maneira que se execute o PCA sem o desperdício de variância dessas componentes cinemáticas, mas este experimento ainda não foi realizado.
Também é importante lembrar que existem métodos mais eficazes e direcionados para a determinação de tais propriedades
cinemáticas. Nas galáxias presentes no CALIFA não poderia ser diferente, portanto os espectros aparecem com linhas
deslocadas para o azul ({\em blue-shifted}) ou para o vermelho ({\em red-shifted}) dependendo da velocidade de rotação
projetada. A disperção de velocidades em cada ponto da galáxia também pode causar alargamento ou estreitamento das
linhas. {\bf\ojo CONFUSO....}

\begin{figure}
    \includegraphics[width=0.9\textwidth]{figuras/K0277-tomo-syn-norm.pdf}
    \caption[Tomogramas de 1 a 5 da gal\'axia NGC 2916 - $F_{syn} / F_{\lambda 5365}$ .]
    {Cinco primeiras PCs (e seus respectivos tomogramas) resultantes da Tomografia PCA aplicado aos espectros
    sintéticos normalizados da galáxia NGC 2916.}
    \label{fig:PCAaplic:K277tomofsynnorm}
\end{figure}





\section{PCA logarítmica}
\label{sec:PCAaplic:PCAlog}

Um outro experimento que realizamos foi calcular o PCA sobre os logarítmos dos espectros.  Essa variante da analise pode ser vista como uma mera curiosidade matematica, uma experiencia para ver o que se obtem. No entanto, ela foi inspirada em um motivo físico: É razoavel esperar que utilizando o log possamos ressaltar efeitos devidos a gradientes de poeira, já que a extincao que atua de forma multiplicativa sobre os espectros.

Para entender melhor voltemos ao caso hipotetico de uma galaxia na qual as populacoes estelares sao iguais em todas partes e na qual as estrelas estao todas paradas, de modo que nao ha deslocamentos nnem modulacoes em $\lambda$ devido a efeitos de cinematica. Neste caso, as variacoes de $\textbf{F}{}_{z \lambda}$ entre as diferentes zonas vem tanto do efeito de amplitude (brilho superficial) já estudado como de diferencas na profudindade optica da poeira $\tau_{z \lambda}$. O efeito global de amplitude pode ser removido com nosso esquema de normalizacao, mas as variacoes de $\tau_{z \lambda}$ de zona a zona afetam a forma do espectro observado.

O efeito da poeira é atenuar um fluxo intrinseco $F^i$ por um fato $e^{-\tau}$, produzindo um fluxo observado (transmitido) $F^o$. Escrevendo $F^o = F^i e^{-\tau}$ para um certo $\lambda$ e normalizando-o pelo comprimento de onda de normalizacao $\lambda_N$ obtemos


\begin{equation}
\frac{\textbf{F}^o_{z \lambda} }{\textbf{F}^o_{z \lambda_N} } =
 \frac{\textbf{F}^i_{z \lambda}  }{\textbf{F}^i_{z \lambda_N} } 
 \frac{ e^{-\tau_{z \lambda}} }{ e^{-\tau_{z \lambda_N}}}
\end{equation}

Denotando os espectros normalizados por $O_\lambda$ e assumindo que a lei de extincao é a mesma em todas zonas

\begin{equation}
\textbf{O}^o_{z \lambda} = \textbf{O}^i_{z \lambda}
 e^{-\tau_{z V} (q_\lambda - q_{\lambda_N})}
\end{equation}

\noindent onde $q_\lambda \equiv \tau_\lambda / \tau_V$ regula o grau de avermelhamento. Tirando o ln

\begin{equation}
\ln \textbf{O}^o_{z \lambda} = \ln \textbf{O}^i_{z \lambda}
 -\tau_{z V} (q_\lambda - q_{\lambda_N})
\end{equation}


Se $\textbf{O}^i_{z \lambda}$ é igual para todas zonas, como é o caso em nossa galáxia hipotética, fica claro que toda variancia em 
$\ln \textbf{O}^o_{z \lambda}$ estará associada a variancia de $\tau_{z V}$ de zona-a-zona, e a tomografia PCA mapeará essa variancia. 
Sem as operacoes de normalizacao e logaritmo, esse efeito apareceria misturado com outros. 

CALIFA 277, a galáxia que temos usado como exemplo ao longo desse capitulo, contem pouca poeira, nao sendo portanto um bom caso para aplicacao dessa variante. Exemplificaremos os resultados da PCA logaritimica no capitulo 5 ao estudarmos os sistemas em fusao ({\em mergers}) Arp 220 e NGC ????, ricos em poeira.



\section{Comparando as PCs com o \STARLIGHT: engenharia reversa}
\label{sec:PCAaplic:EngRev}

Descobrir o sentido físico de cada PC não é tarefa fácil. Como comentamos anteriormente, o PCA te dá a resposta, mas
você geralmente não sabe qual a pergunta. Uma forma de buscar sentido físico nas componentes é analisar as correlações
com propriedades físicas da galáxia. 

Com o resultado da síntese de populações estelares obtido pelo \starlight e
organizado pelo PyCASSO fica simples correlacionarmos os dados na base gerada pelo PCA. Nas Figuras
\ref{fig:PCAaplic:K0277correfobsnorm} e \ref{fig:PCAaplic:K0277correfsynorm} vemos as correlações entre algumas propriedades 
físicas ($\langle\log t \rangle$, $\log \langle Z \rangle$, $A_V$, $v_{\star}$, $\sigma_{\star}$) e o peso de cada PC por zona (tomograma
versus propriedade física) para CALIFA 277. O número presente em cada gráfico é o coeficiente de correlação por
{\em rank} de Spearman. Escolhemos o coeficiente de Spearman pois este é aparamétrico e, ao contrário do coeficiente de
Pearson, não pressupõe nenhuma relação linear. Em cada coluna temos uma quantidade física e na última coluna temos o autovetor representado por aqueles pontos.

Tanto
para os dados observados quanto sintéticos vemos que a primeira PC representa basicamente um fator de idade. Possui
correlação também com a metalicidade. A segunda PC também em ambos os casos representa uma forte correlação com a
velocidade estelar, refletindo a curva de rotcao da galáxia. A terceira PC com os dados observados parece ser um bom sensor para o padrão global de extinção
($A_V$). Essa PC parece se dissolver em duas no caso do espectro sintético (PC3 e 4). As PC4 e PC5 no caso observado não
parecem ter correlação com nenhum desses parâmetros físicos comparados, salvo pequenas correlações com $A_V$ e $v_\star$
na PC5. Essa última, para o caso sintético, parece estar correlacionada com $A_V$ também, mas de forma mais fraca.
Também possui um padrão de correlação com $v_\star$. Por fim, a PC6 no caso observado parece ter uma mistura de idade,
metalicidade, extinção e com a disperção de velocidades. Para o caso sintético existe uma correlação com a metalicidade
e a dispersão de velocidades. Um padrão de rotação também aparece mas com um coeficiente de spearmann baixo.

Outra forma de se observar essas correlações é graficando os dados nos eixos das PCs, colorindo os pontos por
determinado parâmetro físico. No Apêndice estarão todos os gráficos com todos os parâmetros físicos. Como exemplo aqui
deixamos as Figuras \ref{fig:PCAaplic:K0277correfobsnormPCvsPC:AV} e \ref{fig:PCAaplic:K0277correfsynnormPCvsPC:AV} que mostram
as PCs para o caso observado e sintético, coloridos pelo $A_V$.

\begin{figure}
    \includegraphics[width=1.3\textwidth, angle=-90]{figuras/K0277-correl-f_obs_norm-PCvsPhys.pdf}
	\caption[Correlações PCs vs. par\^ametros f\'isicos - $F_{obs}$ norm.]
    {Correlações entre os pesos por zona das seis primeiras PCs do PCA feito para o cubo com os dados observados
    normalizados e cinco parâmetros físicos.q Pela ordem de colunas da esquerda para direita temos $\log$ t, $\log$ $Z /
    Z_{\odot}$, $A_V$, $v_{\star}$, $\sigma_{\star}$. Na última coluna temos o autoespectro para ajudar na visualização.
    A linha em vermelho no gráfico do autoespectro serve para identificar o zero. O número dentro de cada gráfico é o
    coeficiente de correlação de Spearman.}
    \label{fig:PCAaplic:K0277correfobsnorm}
\end{figure}

\begin{figure}
    \includegraphics[width=1.3\textwidth, angle=-90]{figuras/K0277-correl-f_syn_norm-PCvsPhys.pdf}
	\caption[Correlações PCs vs. par\^ametros f\'isicos - $F_{syn}$ norm.]
    {Correlações entre os pesos por zona das seis primeiras PCs do PCA feito para o cubo com os dados sintéticos
    normalizados e cinco parâmetros físicos.q Pela ordem de colunas da esquerda para direita temos $\log$ t, $\log$ $Z /
    Z_{\odot}$, $A_V$, $v_{\star}$, $\sigma_{\star}$. Na última coluna temos o autoespectro para ajudar na visualização.
    A linha em vermelho no gráfico do autoespectro serve para identificar o zero. O número dentro de cada gráfico é o
    coeficiente de correlação de Spearman.}
    \label{fig:PCAaplic:K0277correfsynorm}
\end{figure}

\begin{figure}
	\includegraphics[width=1.4\textwidth, angle=-90]{figuras/K0277-f_obs_norm-corre_PCxPC_AV.pdf}
	\caption[Dados no espaço das PCs vs AV- $F_{obs}$ norm.]
    {Dados das zonas no espaço das PCs (com espectros observados) coloridos pela extinção ($A_V$).}
    \label{fig:PCAaplic:K0277correfobsnormPCvsPC:AV}	
\end{figure}

\begin{figure}
	\includegraphics[width=1.4\textwidth, angle=-90]{figuras/K0277-f_syn_norm-corre_PCxPC_AV.pdf}
	\caption[Dados no espaço das PCs vs AV- $F_{syn}$ norm.]
    {Dados das zonas no espaço das PCs (com espectros sintéticos) coloridos pela extinção ($A_V$).}
    \label{fig:PCAaplic:K0277correfsynnormPCvsPC:AV}	
\end{figure}

A Figura \ref{fig:PCAaplic:K0277correfobs} mostra também o efeito que a falta de normalização faz com o resultado do PCA.
Esse fator de amplitude que não é removido pela normalização se concentra principalmente na primeira PC, mas ainda
deixa vestígios se misturando com as outras PCs fazendo com que haja correlação entre cada PCs e quase todos os parâmetros físicos. Por mais esse motivo, deixaremos a PCA dos espectros não normalizados de lado no restante desse trabalho.

\begin{figure}
    \includegraphics[width=1.3\textwidth, angle=-90]{figuras/K0277-correl-f_obs-PCvsPhys.pdf}
	\caption[Correlações PCs vs. par\^ametros f\'isicos - $F_{obs}$.]
    {Correlações entre os pesos por zona das seis primeiras PCs do PCA feito para o cubo com os dados observados e cinco
    parâmetros físicos.q Pela ordem de colunas da esquerda para direita temos $\log$ t, $\log$ $Z / Z_{\odot}$, $A_V$,
    $v_{\star}$, $\sigma_{\star}$. Na última coluna temos o autoespectro para ajudar na visualização. A linha em
    vermelho no gráfico do autoespectro serve para identificar o zero.}
    \label{fig:PCAaplic:K0277correfobs}
\end{figure}


%\ldots \dots \ldots \ldots
%
%Caso as grandezas físicas analisadas em uma galáxia fossem não-correlacionadas seria muito mais fácil fazer uma comparação entre PCs e grandezas físicas, mas geralmente não é o que ocorre. Para colhermos informações sobre os objetos no céu só temos duas formas: através imagens ou de espectros. Em ambas formas, diferentes efeitos físicos causam efeitos semelhantes nas cores (imagem) ou nos espectros. Esses efeitos fazem com que alguns dos parâmetros físicos fiquem extremamente correlacionados.\ojo \citneed \textcolor{red}{Gostaria de adicionar aqui algumas referências como alguns dos papers do SEAGal/Starlight que fala sobre as degenerescencias de idade e metalicidade e algumas outras grandezas correlacionadas. Coloquei essa parte aqui pensando naquela crítica que o cara me fez lá no México sobre as grandezas em astrofísica serem extremamente correlacionadas, mas não sei até que ponto é legal comentar sobre isso visto que estamos analisando as coisas via correlação\ldots talvez essa discussão possa ficar mais pro Cap 6 nas conclusões.}.

\ldots \dots \ldots \ldots
                                                                                                                                                                                                                                                                               
\textcolor{red}{DAQUI PRA BAIXO AINDA É O ESQUELETO}
\ojo A Filosofia desse capítulo é aprender/testar como operar o PCA para que ele
reflita isso ou aquilo... descrevemos uma série de experimentos nesse sentido.

Preprocessamentos e diferentes tipos de PCAs com ou sem linhas, diferentes
faixas espectrais, com dados normalizados ou não. (importante!!!)

Vamos nos limitar a no-emission lines analysis, descrever a máscara de linhas de
emissão, etc. Isso para facilitar a coisa e pq queremos correlar o resultado do
PCA com os dados do Starlight (PyCASSO).

Simulações para ajudar a decifrar os resultador, população jovem + velha +
modelo de distribuição espacial - ver efeitos de estratégias de
preprocessamento.

Correlacionando os resultados do PCA com as propriedades do starlight (tipo de
engenharia reversa)

Linhas telúricas - remover ou não... bad pixels... mascarar ou não linhas de
emissão