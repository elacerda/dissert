%%%%%%%%%%%%%%%%%%%%%%%%%%%%%%%%%%%%%%%%%%%%%%%%%%%%%%%%%%%%%%%%%
% Dissertacao de Mestrado / Dept Fisica, CFM, UFSC              %
% Lacerda@UFSC - 2013                                           %
%%%%%%%%%%%%%%%%%%%%%%%%%%%%%%%%%%%%%%%%%%%%%%%%%%%%%%%%%%%%%%%%%


%:::::::::::::::::::::::::::::::::::::::::::::::::::::::::::::::%
%                                                               %
%                          Capítulo 4                           %
%                                                               %
%:::::::::::::::::::::::::::::::::::::::::::::::::::::::::::::::%

%***************************************************************%
%                                                               %
%                       Testes com PCA                          %
%                                                               %
%***************************************************************%

\chapter{PCA nos espectros}
\label{sec:UsoPCA}

O uso de métodos estatísticos já se estende por séculos em praticamente todas
(senão todas) as áreas de conhecimento. Esse fato cria uma necessidade de que
existam cada vez mais estudos sobre estudos, ou {\em metaestudos}\footnote{Em
alusão a metadados, que são dados sobre dados.}. Precisamos saber de que maneira
os pré-processamentos de nossa amostra afetam os dados e, principalmente, o
resultado após a aplicação de determinada técnica, para que dessa forma, o
desenvolvimento não se torne uma ``caixa preta'' inacessível.

\section{Pré-processamento dos cubos}
\label{sec:UsoPCA:PCAlidades}

Como o PCA é uma técnica em qual calcula-se os eixos que, através da variância,
melhor expandem sua base de dados, é natural que qualquer pré-processamento que
altere a variância dos dados resultará num conjunto diferente de PCs. Estudando
os cubos de dados do CALIFA estamos buscando variâncias espaciais nos espectros.
Durante nossas investigações fizemos uma série de testes com pré-processamentos
nos espectros, mas dois deles são constantes em todos os estudos. Primeiro todos
os espectros são limitados ao intervalo de $3800$ a $6850$ \AA. Após essa
limitação, são removidas todas as linhas telúricas\footnote{Linhas de absorção
referentes à atmosfera.} e então é feito uma estatística com todos os {\em bad
pixels} de cada cubo e todos aqueles que estão presentes em $90\%$ dos espectros
são removídos. Todas as informações de {\em flags}\footnote{Marcações.} em {\em
bad pixels} e linhas telúricas estão contidas em um pipeline de
pré-processamento dos cubos de dados do CALIFA chamado {\sc qbick}. Esse
pipeline, além disso, prepara os dados para o PyCASSO. Todas as informações de
erros, {\em flags}, reamostragem do {\sc qbick} são herdadas pelo PyCASSO.
\fixme \textcolor{red}{ARRUMAR CÓDIGO PCALIFA E GERAR GRÁFICOS OU TABELA}

\subsection{Normalização}
\label{sec:UsoPCA:PCAlidades:norm}

Após esses dois pré-processamentos constantes em todos os cubos podemos fazer
uma normalização nos espectros. No caso do CALIFA, o cubo de espectros inclui um
{\em FoV} que abrange praticamente toda a galáxia. Isso gera uma variância
indevida entres as zonas devido a luminosidade mais intensa nas zonas centrais
da galáxia em comparação com as mais afastadas. Indevidas pois não trazem
informação nova para a nossa análise. Cada espectro é então normalizado pelo seu
fluxo em $5635$ \AA. Podemos ver um exemplo da diferença nas PCs na figura
\fixme \textcolor{red}{GERAR GRÁFICOS}.

\subsection{Fluxos observados e sintéticos}
\label{sec:UsoPCA:PCAlidades:flux}

\fixme \textcolor{red}{Isso não é um pré-processamento\ldots aonde colocar???}

Com o PyCASSO, temos o resultado da síntese de populações estelares já
organizado para as galáxias do CALIFA. Com isso realizamos o PCA no cubo de
espectros observados e no de espectros sintéticos, com e sem normalização. A
grande diferença é que nos espectros sintéticos estão contidas apenas as
informações sobre populações estelares\footnote{Suavização, correções por
poeira e cinemática também são feitas nos espectros no processo de síntese. Mais
detalhes em \citet{CidFernandes2005}}.

\subsection{Linhas de emissão e intervalos específicos em comprimento de onda}
\label{sec:UsoPCA:PCAlidades:emlin}

Nos espectros, além dos {\em bad pixels} e linhas telúricas, podemos mascarar
regiões desnecessárias para determinada investigação científica. Nosso foco é o
estudo das populações estelares, portanto necessitamos que as linhas de emissão,
geralmente associadas ao gás presente nas galáxias \fixme sejam removidas do
espectro. Dessa forma podemos fazer correlações entre os resultados do PCA e as
propriedades físicas obtidas pela síntese.

Podemos também executar o PCA apenas em intervalos específicos do espectro. Isso
pode ser feito de várias formas, por exemplo utilizando todos os pontos do(s)
intervalo(s) em $\lambda$, ou utilizando apenas o fluxo integrado, ou apenas
larguras equivalentes de linhas, ou FWHM das linhas. Um exemplo aplicado é fazer
o PCA apenas das regiões que abrangem $\oIII/\Hbeta$ em conjunto com
$\nII/\Halpha$ ou então $\mathrm{H}\delta$ e D$4000$ para estudar a variância
espacil desses ratios.

Nas nossas análises no próximo capítulo \ref{sec:result} são mascaradas também
todas as regiões que são removidas na síntese ($\mathrm{H}\epsilon$: de $3960$ a
$3980$ \AA; $\mathrm{H}\delta$: de $4092$ a $4112$ \AA; $\mathrm{H}\gamma$: de
$4330$ a $4350$ \AA; \Hbeta: de $4848$ a $4874$ \AA; \oIII: de $4940$ a $5028$
\AA; $\mathrm{He\,\textsc{i}}$ e $\mathrm{NaD}$: de $5866$ a $5916$ \AA; \Halpha
e \nII: de $6528$ a $6608$ \AA; $\mathrm{S\,\textsc{ii}}$: de $6696$ a $6752$
\AA).

\ldots \dots \ldots \ldots

\textcolor{red}{DAQUI PRA BAIXO AINDA É O ESQUELETO}
\ojo A Filosofia desse capítulo é aprender/testar como operar o PCA para que ele
reflita isso ou aquilo... descrevemos uma série de experimentos nesse sentido.

Preprocessamentos e diferentes tipos de PCAs com ou sem linhas, diferentes
faixas espectrais, com dados normalizados ou não. (importante!!!)

Vamos nos limitar a no-emission lines analysis, descrever a máscara de linhas de
emissão, etc. Isso para facilitar a coisa e pq queremos correlar o resultado do
PCA com os dados do Starlight (PyCASSO).

Simulações para ajudar a decifrar os resultador, população jovem + velha +
modelo de distribuição espacial - ver efeitos de estratégias de
preprocessamento.

Correlacionando os resultados do PCA com as propriedades do starlight (tipo de
engenharia reversa)

Linhas telúricas - remover ou não... bad pixels... mascarar ou não linhas de
emissão