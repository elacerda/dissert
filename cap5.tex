%%%%%%%%%%%%%%%%%%%%%%%%%%%%%%%%%%%%%%%%%%%%%%%%%%%%%%%%%%%%%%%%%
% Dissertacao de Mestrado / Dept Fisica, CFM, UFSC              %
% Lacerda@UFSC - 2013                                           %
%%%%%%%%%%%%%%%%%%%%%%%%%%%%%%%%%%%%%%%%%%%%%%%%%%%%%%%%%%%%%%%%%

%:::::::::::::::::::::::::::::::::::::::::::::::::::::::::::::::%
%                                                               %
%                          Capítulo 5                           %
%                                                               %
%:::::::::::::::::::::::::::::::::::::::::::::::::::::::::::::::%

%***************************************************************%
%                                                               %
%                          Resultados                           %
%                                                               %
%***************************************************************%

\chapter{Aplicando a Tomografia PCA em 8 galáxias do CALIFA}
\label{sec:result}

Com a possibilidade da aplicação de técnicas estatísticas entre as populações estelares distribuídas em uma mesma
galáxia, abarcando, em campo de visão, toda galáxia, estamos munidos para a primeira exploração cientifica usando
Tomografia PCA em galáxias do CALIFA. No capítulo passado mostramos os efeitos nos resultados do PCA para diferentes
configurações do conjunto inicial de espectros. Nesse, aplicaremos a Tomografia PCA nas galáxias presentes na Tabela
\ref{tab:amostraGalaxias} de maneira exploratória, para que possamos identificar determinados padrões de variâncias
nos dados, correlacionando as PCs com determinadas propriedades fisicas de populações estelares.

\ojo Em alguns casos podemos encontrar até problemas que passaram pelo processo de redução, marcação dos píxels
problemáticos, subtração de poeira, em determinado espectro ou cubo.

\begin{table}
	\caption[Relação de galáxias do CALIFA usadas neste trabalho.]
	{}
	\begin{tabular}{l l l r r r}
		Nome da galáxia & CALIFA ID & {\em Hubble Type} & log Mass $M_\odot$ & redshift & $N_z$ \\
		\midrule
		NGC 0001 & K0008 & Sbc & $11.00$ & $0.01515$ & $1132$ \\
		NGC 0776 & K0073 & Sb  & $11.19$ & $0.01640$ & $1733$ \\
		NGC 1167 & K0119 & S0  & $11.47$ & $0.01645$ & $1879$ \\
		NGC 2623 & K0213 & Scd & $10.74$ & $0.01847$ & $561$  \\
		NGC 2916 & K0277 & Sbc & $10.83$ & $0.01244$ & $1638$ \\
		NGC 4210 & K0518 & Sb  & $10.49$ & $0.00906$ & $1938$ \\
		ARP 220  & K0802 & Sd  & $11.15$ & $0.01814$ & $1157$ \\
		NGC 6515 & K0864 & E3  & $11.42$ & $0.02285$ & $887$  \\
	\end{tabular}
	\label{tab:amostraGalaxias}
\end{table}

\ojo
Aplicando PCA e Tomografia PCA para 4 galáxias espirais, 2 elipticas e 2 mergers.

% End of this chapter
