%%%%%%%%%%%%%%%%%%%%%%%%%%%%%%%%%%%%%%%%%%%%%%%%%%%%%%%%%%%%%%%%%
% Dissertacao de Mestrado / Dept Fisica, CFM, UFSC              %
% Lacerda@UFSC - 2013                                           %
%%%%%%%%%%%%%%%%%%%%%%%%%%%%%%%%%%%%%%%%%%%%%%%%%%%%%%%%%%%%%%%%%

%:::::::::::::::::::::::::::::::::::::::::::::::::::::::::::::::%
%                                                               %
%                          Capítulo 5                           %
%                                                               %
%:::::::::::::::::::::::::::::::::::::::::::::::::::::::::::::::%

%***************************************************************%
%                                                               %
%                          Resultados                           %
%                                                               %
%***************************************************************%

\chapter{Aplicando a Tomografia PCA em 8 galáxias do CALIFA}
\label{sec:result}

Com a possibilidade da aplicação de técnicas estatísticas entre as populações estelares distribuídas em uma mesma
galáxia, abarcando, em campo de visão, toda galáxia, estamos munidos para a primeira exploração científica usando
Tomografia PCA em galáxias do CALIFA. No capítulo passado mostramos os efeitos nos resultados do PCA para diferentes
configurações do conjunto inicial de espectros. Nesse, aplicaremos a Tomografia PCA em algumas galáxias de maneira
exploratória, para que possamos identificar determinados padrões de variâncias nos dados. Por fim usaremos de nossa
engenharia reversa, correlacionando as PCs com determinadas propriedades fisicas de populações estelares obtidas pela
síntese.

\ojo Em alguns casos podemos encontrar até problemas que passaram pelo processo de redução, marcação dos píxels
problemáticos, subtração de poeira, em determinado espectro ou cubo.

\section{Apresentação das galáxias}
\label{sec:result:apres}

\begin{table}
	\caption[Relação de galáxias do CALIFA usadas neste trabalho.]
	{}
	\begin{tabular}{l c c c c r}
		Nome da galáxia & CALIFA ID & {\em Hubble Type} & log Mass [$M_\odot$] & redshift & $N_z$ \\
		\midrule
		NGC 0001 & K0008 & Sbc & 11.00 & 0.01515 & 1132 \\
		NGC 2916 & K0277 & Sbc & 10.83 & 0.01244 & 1638 \\
		NGC 0776 & K0073 & Sb  & 11.19 & 0.01640 & 1733 \\
		NGC 4210 & K0518 & Sb  & 10.49 & 0.00906 & 1938 \\
		NGC 1167 & K0119 & S0  & 11.47 & 0.01645 & 1879 \\
		NGC 6515 & K0864 & E3  & 11.42 & 0.02285 & 887  \\
		NGC 2623 & K0213 & Scd & 10.74 & 0.01847 & 561  \\
		ARP 220  & K0802 & Sd  & 11.15 & 0.01814 & 1157 \\
	\end{tabular}
	\label{tab:amostraGalaxias}
\end{table}

\section{Galáxias espirais}
\label{sec:result:spirals}

Dentre as galáxias já observadas pelo projeto CALIFA, escolhemos quatro espirais (NGC 0001, NGC 2916, NGC 0776,
NGC4210), de modo que busquemos a existência de um padrão nos resultados das PCs. Omitimos aqui as imagens referentes a
análise da galáxia NGC 2916 pois já se encontram no capitulo anterior.

\subsection{NGC 0001 - CALIFA 8}

\begin{figure}
    \includegraphics[height=0.33\textheight]{figuras/K0008-screetest.pdf}
    \caption[Scree test comparativo entre 3 PCAs - NGC 0001.]
    {Scree test para 3 análises PCA da galáxia NGC 0001 (CALIFA 8). Com marcações de triângulos vemos as PCs
    resultantes do PCA com os espectros observados normalizados. As variâncias das PCs marcadas com estrela representam
    o PCA com os espectros sintéticos normalizados. Para comparação plotamos as PCs do caso sem normalização usando
    linha pontilhada.}
    \label{fig:K0008scree}
\end{figure}

\begin{figure}
    \includegraphics[width=0.85\textwidth]{figuras/K0008-tomo-obs-norm.pdf}
    \caption[Tomogramas de 1 a 5 para o cubo $F_{obs}$ norm. - NGC 0001.]
    {Cinco primeiras PCs (e seus respectivos tomogramas) resultantes da Tomografia PCA aplicado aos espectros
    observados com normalização da galáxia NGC 0001.}
    \label{fig:K0008tomofobsnorm}
\end{figure}

\begin{figure}
    \includegraphics[width=0.85\textwidth]{figuras/K0008-tomo-syn-norm.pdf}
    \caption[Tomogramas de 1 a 5 para o cubo $F_{syn}$ norm. - NGC 0001.]
    {Cinco primeiras PCs (e seus respectivos tomogramas) resultantes da Tomografia PCA aplicado aos espectros
    sintéticos com normalização da galáxia NGC 0001.}
    \label{fig:K0008tomofsynnorm}
\end{figure}

\begin{figure}
    \includegraphics[width=1.3\textwidth, angle=-90]{figuras/K0008-correl-f_obs_norm-PCvsPhys.pdf}
	\caption[Correlações PCs vs. par\^ametros f\'isicos - $F_{obs}$ norm. - NGC 0001]
    {Correlações entre os pesos por zona das seis primeiras PCs do PCA feito para o cubo com os espectros observados
    normalizados e cinco parâmetros físicos.q Pela ordem de colunas da esquerda para direita temos $\log$ t, $\log$ $Z /
    Z_{\odot}$, $A_V$, $v_{\star}$, $\sigma_{\star}$. Na última coluna temos o autoespectro para ajudar na visualização.
    A linha em vermelho no gráfico do autoespectro serve para identificar o zero. O número dentro de cada gráfico é o
    coeficiente de correlação de Spearman.}
    \label{fig:K0008correfobsnorm}
\end{figure}

\begin{figure}
    \includegraphics[width=1.3\textwidth, angle=-90]{figuras/K0008-correl-f_syn_norm-PCvsPhys.pdf}
	\caption[Correlações PCs vs. par\^ametros f\'isicos - $F_{syn}$ norm. - NGC 0001]
    {Correlações entre os pesos por zona das seis primeiras PCs do PCA feito para o cubo com os espectros sintéticos
    normalizados e cinco parâmetros físicos.q Pela ordem de colunas da esquerda para direita temos $\log$ t, $\log$ $Z /
    Z_{\odot}$, $A_V$, $v_{\star}$, $\sigma_{\star}$. Na última coluna temos o autoespectro para ajudar na visualização.
    A linha em vermelho no gráfico do autoespectro serve para identificar o zero. O número dentro de cada gráfico é o
    coeficiente de correlação de Spearman.}
    \label{fig:K0008correfsynnorm}
\end{figure}

\subsection{NGC 0776 - CALIFA 73}

\begin{figure}
    \includegraphics[height=0.33\textheight]{figuras/K0073-screetest.pdf}
    \caption[Scree test comparativo entre 3 PCAs - NGC 0776.]
    {Igual a Figura \ref{fig:K0008scree} para a galáxia NGC 0776.}
    \label{fig:K0073scree}
\end{figure}

\begin{figure}
    \includegraphics[width=0.85\textwidth]{figuras/K0073-tomo-obs-norm.pdf}
    \caption[Tomogramas de 1 a 5 para o cubo $F_{obs}$ norm. - NGC 0776.]
    {Igual a Figura \ref{fig:K0008tomofobsnorm} para a galáxia NGC 0776.}
    \label{fig:K0073tomofobsnorm}
\end{figure}

\begin{figure}
    \includegraphics[width=0.85\textwidth]{figuras/K0073-tomo-syn-norm.pdf}
    \caption[Tomogramas de 1 a 5 para o cubo $F_{syn}$ norm. - NGC 0776.]
    {Igual a Figura \ref{fig:K0008tomofsynnorm} para a galáxia NGC 0776.}
    \label{fig:K0073tomofsynnorm}
\end{figure}

\begin{figure}
    \includegraphics[width=1.3\textwidth, angle=-90]{figuras/K0073-correl-f_obs_norm-PCvsPhys.pdf}
	\caption[Correlações PCs vs. par\^ametros f\'isicos - $F_{obs}$ norm. - NGC 0001]
	{Igual a Figura \ref{fig:K0008correfobsnorm} para a galáxia NGC 0776.}
    \label{fig:K0073correfobsnorm}
\end{figure}

\begin{figure}
    \includegraphics[width=1.3\textwidth, angle=-90]{figuras/K0073-correl-f_syn_norm-PCvsPhys.pdf}
	\caption[Correlações PCs vs. par\^ametros f\'isicos - $F_{syn}$ norm. - NGC 0001]
	{Igual a Figura \ref{fig:K0008correfsynnorm} para a galáxia NGC 0776.}
    \label{fig:K0073correfsynnorm}
\end{figure}


\subsection{NGC 4210 - CALIFA 518}

\begin{figure}
    \includegraphics[height=0.33\textheight]{figuras/K0518-screetest.pdf}
    \caption[Scree test comparativo entre 3 PCAs - NGC 4210.]
	{Igual a Figura \ref{fig:K0008scree} para a galáxia NGC 4210.}
    \label{fig:K00518scree}
\end{figure}

\begin{figure}
    \includegraphics[width=0.85\textwidth]{figuras/K0518-tomo-obs-norm.pdf}
    \caption[Tomogramas de 1 a 5 para o cubo $F_{obs}$ norm. - NGC 4210.]
    {Igual a Figura \ref{fig:K0008tomofobsnorm} para a galáxia NGC 4210.}
    \label{fig:K0518tomofobsnorm}
\end{figure}

\begin{figure}
    \includegraphics[width=0.85\textwidth]{figuras/K0518-tomo-syn-norm.pdf}
    \caption[Tomogramas de 1 a 5 para o cubo $F_{syn}$ norm. - NGC 4210.]
    {Igual a Figura \ref{fig:K0008tomofsynnorm} para a galáxia NGC 4210.}
    \label{fig:K0518tomofsynnorm}
\end{figure}

\begin{figure}
    \includegraphics[width=1.3\textwidth, angle=-90]{figuras/K0518-correl-f_obs_norm-PCvsPhys.pdf}
	\caption[Correlações PCs vs. par\^ametros f\'isicos - $F_{obs}$ norm. - NGC 4210.]
	{Igual a Figura \ref{fig:K0008correfobsnorm} para a galáxia NGC 4210.}
    \label{fig:K0518correfobsnorm}
\end{figure}

\begin{figure}
    \includegraphics[width=1.3\textwidth, angle=-90]{figuras/K0518-correl-f_syn_norm-PCvsPhys.pdf}
	\caption[Correlações PCs vs. par\^ametros f\'isicos - $F_{syn}$ norm. - NGC 4210.]
	{Igual a Figura \ref{fig:K0008correfsynnorm} para a galáxia NGC 4210.}
    \label{fig:K0518correfsynnorm}
\end{figure}

\section{Galáxias elípticas}
\ref{sec:result:elipt}

\subsection{NGC 1167 - CALIFA 119}

\begin{figure}
    \includegraphics[height=0.33\textheight]{figuras/K0119-screetest.pdf}
    \caption[Scree test comparativo entre 3 PCAs - NGC 1167.]
    {Igual a Figura \ref{fig:K0008scree} para a galáxia NGC 1167.}
    \label{fig:K0119scree}
\end{figure}

\begin{figure}
    \includegraphics[width=0.85\textwidth]{figuras/K0119-tomo-obs-norm.pdf}
    \caption[Tomogramas de 1 a 5 para o cubo $F_{obs}$ norm. - NGC 1167.]
    {Igual a Figura \ref{fig:K0008tomofobsnorm} para a galáxia NGC 1167.}
    \label{fig:K0119tomofobsnorm}
\end{figure}

\begin{figure}
    \includegraphics[width=0.85\textwidth]{figuras/K0119-tomo-syn-norm.pdf}
    \caption[Tomogramas de 1 a 5 para o cubo $F_{syn}$ norm. - NGC 1167.]
    {Igual a Figura \ref{fig:K0008tomofsynnorm} para a galáxia NGC 1167.}
    \label{fig:K0119tomofsynnorm}
\end{figure}

\begin{figure}
    \includegraphics[width=1.3\textwidth, angle=-90]{figuras/K0119-correl-f_obs_norm-PCvsPhys.pdf}
	\caption[Correlações PCs vs. par\^ametros f\'isicos - $F_{obs}$ norm. - NGC 1167.]
	{Igual a Figura \ref{fig:K0008correfobsnorm} para a galáxia NGC 1167.}
    \label{fig:K0119correfobsnorm}
\end{figure}

\begin{figure}
    \includegraphics[width=1.3\textwidth, angle=-90]{figuras/K0119-correl-f_syn_norm-PCvsPhys.pdf}
	\caption[Correlações PCs vs. par\^ametros f\'isicos - $F_{syn}$ norm. - NGC 1167.]
	{Igual a Figura \ref{fig:K0008correfsynnorm} para a galáxia NGC 1167.}
    \label{fig:K0119correfsynnorm}
\end{figure}

\subsection{NGC 6515 - CALIFA 864}

\begin{figure}
    \includegraphics[height=0.33\textheight]{figuras/K0864-screetest.pdf}
    \caption[Scree test comparativo entre 3 PCAs - NGC 6515.]
	{Igual a Figura \ref{fig:K0008scree} para a galáxia NGC 6515.}
    \label{fig:K0864scree}
\end{figure}

\begin{figure}
    \includegraphics[width=0.85\textwidth]{figuras/K0864-tomo-obs-norm.pdf}
    \caption[Tomogramas de 1 a 5 para o cubo $F_{obs}$ norm. - NGC 6515.]
    {Igual a Figura \ref{fig:K0008tomofobsnorm} para a galáxia NGC 6515.}
    \label{fig:K0864tomofobsnorm}
\end{figure}

\begin{figure}
    \includegraphics[width=0.85\textwidth]{figuras/K0864-tomo-syn-norm.pdf}
    \caption[Tomogramas de 1 a 5 para o cubo $F_{syn}$ norm. - NGC 6515.]
    {Igual a Figura \ref{fig:K0008tomofsynnorm} para a galáxia NGC 6515.}
    \label{fig:K0864tomofsynnorm}
\end{figure}

\begin{figure}
    \includegraphics[width=1.3\textwidth, angle=-90]{figuras/K0864-correl-f_obs_norm-PCvsPhys.pdf}
	\caption[Correlações PCs vs. par\^ametros f\'isicos - $F_{obs}$ norm. - NGC 6515.]
	{Igual a Figura \ref{fig:K0008correfobsnorm} para a galáxia NGC 6515.}
    \label{fig:K0864correfobsnorm}
\end{figure}

\begin{figure}
    \includegraphics[width=1.3\textwidth, angle=-90]{figuras/K0864-correl-f_syn_norm-PCvsPhys.pdf}
	\caption[Correlações PCs vs. par\^ametros f\'isicos - $F_{syn}$ norm. - NGC 6515.]
	{Igual a Figura \ref{fig:K0008correfsynnorm} para a galáxia NGC 6515.}
    \label{fig:K0864correfsynnorm}
\end{figure}

\section{{\em Mergers}}
\ref{sec:result:mergers}

\subsection{NGC 2623 - CALIFA 213}

\begin{figure}
    \includegraphics[height=0.33\textheight]{figuras/K0213-screetest.pdf}
    \caption[Scree test comparativo entre 3 PCAs - NGC 2623.]
    {Igual a Figura \ref{fig:K0008} para a galáxia NGC 2623.}
    \label{fig:K0213scree}
\end{figure}

\begin{figure}
    \includegraphics[width=0.85\textwidth]{figuras/K0213-tomo-obs-norm.pdf}
    \caption[Tomogramas de 1 a 5 para o cubo $F_{obs}$ norm. - NGC 2623.]
    {Igual a Figura \ref{fig:K0008tomofobsnorm} para a galáxia NGC 2623.}
    \label{fig:K0213tomofobsnorm}
\end{figure}

\begin{figure}
    \includegraphics[width=0.85\textwidth]{figuras/K0213-tomo-syn-norm.pdf}
    \caption[Tomogramas de 1 a 5 para o cubo $F_{syn}$ norm. - NGC 2623.]
    {Igual a Figura \ref{fig:K0008tomofsynnorm} para a galáxia NGC 2623.}
    \label{fig:K0213tomofsynnorm}
\end{figure}

\begin{figure}
    \includegraphics[width=1.3\textwidth, angle=-90]{figuras/K0213-correl-f_obs_norm-PCvsPhys.pdf}
	\caption[Correlações PCs vs. par\^ametros f\'isicos - $F_{obs}$ norm. - NGC 2623.]
	{Igual a Figura \ref{fig:K0008correfobsnorm} para a galáxia NGC 2623.}
    \label{fig:K0213correfobsnorm}
\end{figure}

\begin{figure}
    \includegraphics[width=1.3\textwidth, angle=-90]{figuras/K0213-correl-f_syn_norm-PCvsPhys.pdf}
	\caption[Correlações PCs vs. par\^ametros f\'isicos - $F_{syn}$ norm. - NGC 2623.]
	{Igual a Figura \ref{fig:K0008correfsynnorm} para a galáxia NGC 2623.}
    \label{fig:K0213correfsynnorm}
\end{figure}

\subsection{ARP 220 - CALIFA 802}

\begin{figure}
    \includegraphics[height=0.33\textheight]{figuras/K0802-screetest.pdf}
    \caption[Scree test comparativo entre 3 PCAs - ARP 220.]
	{Igual a Figura \ref{fig:K0008} para a galáxia ARP 220.}
    \label{fig:K0802scree}
\end{figure}

\begin{figure}
    \includegraphics[width=0.85\textwidth]{figuras/K0802-tomo-obs-norm.pdf}
    \caption[Tomogramas de 1 a 5 para o cubo $F_{obs}$ norm. - ARP 220.]
    {Igual a Figura \ref{fig:K0008tomofobsnorm} para a galáxia ARP 220.}
    \label{fig:K0802tomofobsnorm}
\end{figure}

\begin{figure}
    \includegraphics[width=0.85\textwidth]{figuras/K0802-tomo-syn-norm.pdf}
    \caption[Tomogramas de 1 a 5 para o cubo $F_{obs}$ norm. - ARP 220.]
    {Igual a Figura \ref{fig:K0008tomofsynnorm} para a galáxia ARP 220.}
    \label{fig:K0802tomofsynnorm}
\end{figure}

\begin{figure}
    \includegraphics[width=1.3\textwidth, angle=-90]{figuras/K0802-correl-f_obs_norm-PCvsPhys.pdf}
	\caption[Correlações PCs vs. par\^ametros f\'isicos - $F_{obs}$ norm. - ARP 220.]
	{Igual a Figura \ref{fig:K0008correfobsnorm} para a galáxia ARP 220.}
    \label{fig:K0802correfobsnorm}
\end{figure}

\begin{figure}
    \includegraphics[width=1.3\textwidth, angle=-90]{figuras/K0802-correl-f_syn_norm-PCvsPhys.pdf}
	\caption[Correlações PCs vs. par\^ametros f\'isicos - $F_{syn}$ norm. - ARP 220.]
	{Igual a Figura \ref{fig:K0008correfsynnorm} para a galáxia ARP 220.}
    \label{fig:K0802correfsynnorm}
\end{figure}

% End of this chapter
