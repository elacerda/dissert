%%%%%%%%%%%%%%%%%%%%%%%%%%%%%%%%%%%%%%%%%%%%%%%%%%%%%%%%%%%%%%%%%
% Dissertacao de Mestrado / Dept Fisica, CFM, UFSC              %
% Lacerda@UFSC - 2013                                           %
%%%%%%%%%%%%%%%%%%%%%%%%%%%%%%%%%%%%%%%%%%%%%%%%%%%%%%%%%%%%%%%%%

%:::::::::::::::::::::::::::::::::::::::::::::::::::::::::::::::%
%                                                               %
%                          Capítulo 2                           %
%                                                               %
%:::::::::::::::::::::::::::::::::::::::::::::::::::::::::::::::%

%***************************************************************%
%                                                               %
%                      CALIFA & PyCASSO                         %
%                                                               %
%***************************************************************%

\chapter{O projeto CALIFA e o {\em pipeline} PyCASSO}
\label{sec:CALePyC}

As observações do universo modificaram completamente o nosso modo de viver e se
compreender. Aprendemos a contar os dias, desenvolvemos um sistema de meses,
estações do ano, movimentos das marés, entre outras coisas que já são parte do
senso comum, mas que um dia foram ciência de ponta (mesmo antes de serem
consideradas assim). É assim que surge o CALIFA: um projeto que está
modificando nossa maneira de ver e pensar as galáxias no nosso universo de
forma que entendamos melhor a nossa também.

Com a massiva quantidade de dados obtidos, resultado direto de um projeto de
ciência de ponta, vem também a dificuldade da interpretação dos dados. No caso
do CALIFA, através da {\em pipeline} PyCASSO \citep{CidFernandes2013a}, a
programação investigativa se torna simples e ao mesmo tempo robusta,
facilitando a construçao de todo tipo de gráfico, tabela, correlação, etc.

%***************************************************************%
%                                                               %
%                            CALIFA                             %
%                                                               %
%***************************************************************%

\section{O survey CALIFA}
\label{sec:CALePyC:Apresent}

\ldots
Descrição básica. Figuras dos papers para exemplos de espectros, imagens e 
produtos do PyCASSO. Citar que usamos o COMBO (V500 + V1200).
\ldots

No sul da Espanha, mais precisamente em {\em Sierra de Los Filabres}
(Andalucía), está situado o germano-espânico {\em Calar Alto Observatory}. O
projeto CALIFA está sendo possível através do uso de um dos seus $3$ telescópios
(o maior deles, de $3.5$m) e terá ao todo $\sim 600$ objetos ao longo de $250$
noites de observação. Neste telescópio está instalado o equipamento Potsdam
Multi Aperture Spectrograph \citep[PMAS; ][]{Roth2005} no modo PPAK
\citep{Verheijen2004, Kelz2006}. O {\em bundle} de fibras do PPAK consiste em
$382$ fibras \fixme \textcolor{blue}{NECESSITA FIGURAS!!}, das quais, $331$ são
para observação dos objetos, outras $36$ para \fixme \textcolor{blue}{sky
background sample} e outras $15$ para calibração. As {\em science fibers}
($331$) cobrem um {\em FoV} ({\em field-of-view}) hexagonal de $74" X 64"$ que,
através de uma técnica de três pontos de dithering \fixme \citneed é possível
ter $100\%$ de luz entre as espectrofotômetro de campo integrado PMAS/PPAK com
um  de $\sim1.3$ arcsec$^2$ no qual o projeto CALIFA está obtendo seus dados
em, ao todo, $250$ noites de observação.

\fixme \textcolor{blue}{ligação das duas partes acima e abaixo}
 
Esses objetos estão no universo local com $redshifts$ entre $0.005 < z < 0.03$
observados com o , montado em um telescópio no
observatório de Calar Alto, em Granada, na Espanha. Os objetos estão
distribuídos em uma ampla variedade de tipos morfologicos, massa em estrelas,
condições do gas ionizante.

%***************************************************************%
%                                                               %
%                            PyCASSO                            %
%                                                               %
%***************************************************************%

\section{O {\em pipeline} PyCASSO}
\label{sec:CALePyC:PyCASSO}

% End of this chapter
