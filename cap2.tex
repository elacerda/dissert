%%%%%%%%%%%%%%%%%%%%%%%%%%%%%%%%%%%%%%%%%%%%%%%%%%%%%%%%%%%%%%%%%
% Dissertacao de Mestrado / Dept Fisica, CFM, UFSC              %
% Lacerda@UFSC - 2013                                           %
%%%%%%%%%%%%%%%%%%%%%%%%%%%%%%%%%%%%%%%%%%%%%%%%%%%%%%%%%%%%%%%%%

%:::::::::::::::::::::::::::::::::::::::::::::::::::::::::::::::%
%                                                               %
%                          Capítulo 2                           %
%                                                               %
%:::::::::::::::::::::::::::::::::::::::::::::::::::::::::::::::%

%***************************************************************%
%                                                               %
%                      CALIFA & PyCASSO                         %
%                                                               %
%***************************************************************%

\chapter{O projeto CALIFA e o {\em pipeline} PyCASSO}
\label{sec:CALePyC}

Descrição básica. Figuras dos papers para exemplos de espectros, imagens e 
produtos do PyCASSO. Citar que usamos o COMBO (V500 + V1200).

%***************************************************************%
%                                                               %
%                            CALIFA                             %
%                                                               %
%***************************************************************%

\section{O survey CALIFA}
\label{sec:CALePyC:Apresent}

\ldots
Esses objetos estão no universo local com $redshifts$ entre $0.005 < z < 0.03$
observados com um espectrógrafo PMAS/PPak, montado em um telescópio no
observatório de Calar Alto, em Granada, na Espanha. Os objetos estão
distribuídos em uma ampla variedade de tipos morfologicos, massa em estrelas,
condições do gas ionizante.

%***************************************************************%
%                                                               %
%                            PyCASSO                            %
%                                                               %
%***************************************************************%

\section{O {\em pipeline} PyCASSO}
\label{sec:CALePyC:PyCASSO}

% End of this chapter
