%%%%%%%%%%%%%%%%%%%%%%%%%%%%%%%%%%%%%%%%%%%%%%%%%%%%%%%%%%%%%%%%%
% Dissertacao de Mestrado / Dept Fisica, CFM, UFSC              %
% Lacerda@UFSC - 2013                                           %
%%%%%%%%%%%%%%%%%%%%%%%%%%%%%%%%%%%%%%%%%%%%%%%%%%%%%%%%%%%%%%%%%

%:::::::::::::::::::::::::::::::::::::::::::::::::::::::::::::::%
%                                                               %
%                          Capítulo 2                           %
%                                                               %
%:::::::::::::::::::::::::::::::::::::::::::::::::::::::::::::::%

%***************************************************************%
%                                                               %
%                      CALIFA & PyCASSO                         %
%                                                               %
%***************************************************************%

\chapter{O projeto CALIFA e o {\em pipeline} PyCASSO}
\label{sec:CALePyC}

Descrição básica. Figuras dos papers para exemplos de espectros, imagens e 
produtos do PyCASSO. Citar que usamos o COMBO (V500 + V1200).

%***************************************************************%
%                                                               %
%                            CALIFA                             %
%                                                               %
%***************************************************************%

\section{O survey CALIFA}
\label{sec:CALePyC:Apresent}

%***************************************************************%
%                                                               %
%                            PyCASSO                            %
%                                                               %
%***************************************************************%

\section{O {\em pipeline} PyCASSO}
\label{sec:CALePyC:PyCASSO}

% End of this chapter
