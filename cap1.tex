%%%%%%%%%%%%%%%%%%%%%%%%%%%%%%%%%%%%%%%%%%%%%%%%%%%%%%%%%%%%%%%%%
% Dissertacao de Mestrado / Dept Fisica, CFM, UFSC              %
% Lacerda@UFSC - 2013                                           %
%%%%%%%%%%%%%%%%%%%%%%%%%%%%%%%%%%%%%%%%%%%%%%%%%%%%%%%%%%%%%%%%%

%:::::::::::::::::::::::::::::::::::::::::::::::::::::::::::::::%
%                                                               %
%                          Capítulo 1                           %
%                                                               %
%:::::::::::::::::::::::::::::::::::::::::::::::::::::::::::::::%

%***************************************************************%
%                                                               %
%                         Introdução                            %
%                                                               %
%***************************************************************%

\chapter{Introdução}
\label{sec:Intro}

%***************************************************************%
%                                                               %
%              Introdução - Um labirinto de dados               %
%                                                               %
%***************************************************************%

\section{Um labirinto de dados}
\label{sec:Intro:LabData}

Cientistas hoje, em sua maioria, encontram-se perdidos em meio a um labirinto
de informações. Essas são apenas parte de um cenário incompleto, mas não falso,
que chamamos de nosso universo. Com o avanço tecnológico, melhores formas de se
obter informações tornam cada vez mais evidente o surgimento de um {\em
fordismo}\footnote{O termo {\em fordismo} foi criado por Antonio Gramsci em 1922
e é relativo a Henry Ford e o surgimento da produção em massa de automóveis no
início do século XX.} relativo a informações que auxiliam na criação desse
labirinto, mas ao mesmo tempo fomenta a criatividade e curiosidade de
cientistas que se tornam aventureiros na busca da saída. Mas essa fuga é apenas
a primeira etapa assim se tornando parte basilar na formação da pesquisa
científica.

Os primeiros {\em surveys}\footnote{Um {\em survey} astronômico é um
levantamento de informações ou mapeamento de regiões do céu utilizando
telescópios e detetores.} astronômicos surgem com a inata curiosidade do homem
de observar tudo a sua volta e registrar suas observações. O crescimento
dos catálogos é produto direto da evolução dos equipamentos. No início da década de 80
\citep{Huchra1983} \citep{Huchra1988} \citep{DaCosta1988} a astronomia
extragalática entra nesse cenário de produção sistemática em massa de dados. De
lá para cá a quantidade de dados só aumenta e, com a criação dos
{\em mega-surveys} \citep[\SDSS; ][]{York2000} \citep[2dFGRS;][]{Colless1999}
\citep[2MASS;][]{Skrutskie2006} já terminados e alguns ainda começando ou por
terminar \citep[LSST; ][]{Ivezic2008} \citep[J-PAS;][]{Benitez2009}, beira o humanamente
impossível de se avaliar, necessitando assim da ajuda de métodos computacionais.

Com esse crescimento exponencial na quantidade de dados precisamos cada vez mais
de ferramentas matemáticas/estatísticas. No contexto de espectros de galáxias,
uma ferramenta muito útil para síntese espectral é o \starlight, desenvolvido
por \citet{CidFernandes2005}. Hoje, com o uso de painéis de fibras óticas
apontadas para as galáxias temps os {\em surveys} de IFS ({\em Integral Field
Spectrography})  onde passamos a ter, não um, mas dezenas, centenas e até
milhares de espectros por galáxia, ganhando assim uma dimensão a mais nos dados.
Assim temos para cada píxel (duas dimensões espaciais) um espectro (uma dimensão
espectral), formando assim um cubo de {\em spaxels}\footnote{spectral pixels}.
O pioneiro nessa produção em massa de dados é o {\em Calar Alto Legacy Integral
Field spectroscopy Area survey} \citep[CALIFA; ][]{CALIFAPresent2012},
produzindo cerca de mil espectros por galáxia observada. Outros {\em
mega-surveys} IFS estão por vir {\em XXX TODO XXX REFERENCIAS MANGA, SAMI, VENGA.}. 

Com esses cubos de dados em mão, podemos assim executar o \starlight para
cada {\em spaxel} e então obtemos propriedades físicas em função da posição na
galáxia \citep{CidFernandes20131}. Uma ferramenta menos astrofísica, mas não
menos utilizada, é o PCA ({\em Principal Component Analysis}) e no presente
trabalho fazemos useu uso juntamente com uma técnica criada por
\citet{Steiner2009} chamada de Tomografia PCA, onde une-se imageamento e
espectrografia ao mesmo tempo.

\section{A nova ferramenta - Tomografia PCA}
\label{sec:Intro:TomoPCA}

A técnica de Análise de Componentes Principais (PCA) é simples, não-paramétrica
e nos ajuda a extrair informações de conjuntos de dados com muitas variáveis,
reduzindo a dimensionalidade no sentido de encontrar quais são os elementos com
maior variâncias no seu conjunto de dados. No caso de um espectro, temos uma
medida de fluxo para cada intervalo em comprimento de onda. No caso de um cubo
de spaxels, temos as dimensões espaciais também, obtendo assim uma infinidade de
variáveis. O grande problema do PCA é que você tem a resposta, mas não sabe a
pergunta.

Hoje, PCA é utilizado exaustivamente em várias áreas de conhecimento. {\em XXX
TODO XXX referências e exemplos de uso do PCA}.
% http://arxiv.org/abs/1009.4974
% http://arxiv.org/abs/1208.3700
% http://arxiv.org/abs/astro-ph/0010424
% http://arxiv.org/abs/1310.4217

Com a criação da técnica de Tomografia PCA citada anteriormente temos, além da
análise das componentes principais, uma imagem formada pela matriz de
covariancia (as questões matemáticas serão abordadas mais a frente no presente
trabalho). Assim podemos saber o peso, ou relevância, de cada componente
principal (daqui pra frente PC) relacionada a uma posição na galáxia, aliando a
infinidade de informações físicas presentes nos espectros e nas imagens.

\section{Este trabalho}
\label{sec:Intro:ThisWork}

Através da colaboração do Grupo de Astrofísica da Universidade Federal de Santa
Catarina (GAS-UFSC) com o grupo de pesquisadores do projeto CALIFA tive a
oportunidade de trabalhar com os dados de IFS das galáxias observadas por esse
projeto, que ainda está em andamento. O seu primeiro {\em Data Release}
\citep{CALIFADR1} possui $100$ objetos e por volta de $400000$ espectros. Esses
objetos estão no universo local com $redshifts$ entre $0.005 < z < 0.03$
observados com um espectrógrafo PMAS/PPak, montado em um telescópio no
observatório de Calar Alto, em Granada, na Espanha. Os objetos estão
distribuídos em uma ampla variedade de tipos morfologicos, massa em
estrelas, condições do gas ionizante. A previsão é que ao término do projeto
serão observadas até $600$ objetos.



% End of this chapter
