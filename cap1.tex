%%%%%%%%%%%%%%%%%%%%%%%%%%%%%%%%%%%%%%%%%%%%%%%%%%%%%%%%%%%%%%%%%
% Dissertacao de Mestrado / Dept Fisica, CFM, UFSC              %
% Lacerda@UFSC - 2013                                           %
%%%%%%%%%%%%%%%%%%%%%%%%%%%%%%%%%%%%%%%%%%%%%%%%%%%%%%%%%%%%%%%%%

%:::::::::::::::::::::::::::::::::::::::::::::::::::::::::::::::%
%                                                               %
%                          Capítulo 1                           %
%                                                               %
%:::::::::::::::::::::::::::::::::::::::::::::::::::::::::::::::%

%***************************************************************%
%                                                               %
%                         Introdução                            %
%                                                               %
%***************************************************************%

\chapter{Introdução}
\label{sec:Intro}

%***************************************************************%
%                                                               %
%              Introdução - Um labirinto de dados               %
%                                                               %
%***************************************************************%

\section{Um labirinto de dados}
\label{sec:Intro:LabData}

Cientistas hoje, em sua maioria, encontram-se perdidos em meio a um labirinto
de informações. Essas são apenas parte de um cenário incompleto, mas não falso,
que chamamos de nosso universo. Com o avanço tecnológico, melhores formas de se
obter informações tornam cada vez mais evidente o surgimento de um {\em
fordismo}\footnote{O termo {\em fordismo} foi criado por Antonio Gramsci em 1922
e é relativo a Henry Ford e o surgimento da produção em massa de automóveis no
início do século XX.} relativo a informações que auxiliam na criação desse
labirinto, mas ao mesmo tempo fomenta a criatividade e curiosidade de
cientistas que se tornam aventureiros na busca da saída. Mas essa fuga é apenas
a primeira etapa assim se tornando parte basilar na formação da pesquisa
científica.

Os primeiros {\em surveys}\footnote{Um {\em survey} astronômico é um
levantamento de informações ou mapeamento de regiões do céu utilizando
telescópios e detetores.} astronômicos surgem com a inata curiosidade do homem
de observar tudo a s


ua volta e registrar suas observações. O crescimento dos
catálogos é produto direto da evolução dos equipamentos. No início da década de
80 \citep{Huchra1983, Huchra1988, DaCosta1988} a astronomia extragalática
entra nesse cenário de produção sistemática em massa de dados. De lá para cá a
quantidade de dados só aumenta e, com a criação dos {\em mega-surveys}
\citep[\SDSS; ][]{York2000} \citep[2dFGRS;][]{Colless1999}
\citep[2MASS;][]{Skrutskie2006}, alguns já terminados, outros ainda começando
ou por terminar \citep[LSST; ][]{Ivezic2008} \citep[J-PAS;][]{Benitez2009},
estamos à beira do humanamente impossível de se avaliar, necessitando assim a
ajuda de máquinas e métodos computacionais cada vez mais eficientes.

Com esse crescimento exponencial na quantidade de dados, precisamos cada vez
mais de ferramentas matemáticas/estatísticas. No contexto de espectros de galáxias,
uma ferramenta muito útil para síntese espectral é o \starlight, desenvolvido
por \citet{CidFernandes2005}. Hoje, com o uso de painéis de fibras óticas
apontadas para as galáxias temos os {\em surveys} de IFS ({\em Integral Field
Spectroscopy}) onde passamos a obter, não um, mas dezenas, centenas e até
milhares de espectros por galáxia, ganhando assim dimensão espacial também nos
dados. Assim temos para cada píxel (duas dimensões espaciais) um espectro (uma
dimensão espectral), formando assim um cubo de {\em spaxels}\footnote{spectral
pixels}. O pioneiro nessa produção em massa de dados é o {\em Calar Alto Legacy
Integral Field spectroscopy Area
survey\footnote{\url{http://www.caha.es/CALIFA/}}} \citep[CALIFA;
][]{CALIFAPresent2012}, produzindo cerca de mil espectros por galáxia observada.
Outros {\em mega-surveys} IFS estão por vir (veja a seção
\ref{sec:CALePyC:Apresent}).

Com esses cubos de dados em mão, podemos assim executar o \starlight para
cada {\em spaxel} e então obtemos propriedades físicas em função da posição na
galáxia \citep{CidFernandes2013a}. Uma ferramenta menos astrofísica, mas não
menos utilizada, é o PCA ({\em Principal Component Analysis}) e no presente
trabalho fazemos seu uso juntamente com uma técnica criada por
\citet{Steiner2009} chamada de Tomografia PCA, onde une-se imageamento e
espectrografia ao mesmo tempo.

\section{A nova ferramenta - Tomografia PCA}
\label{sec:Intro:TomoPCA}

A técnica de Análise de Componentes Principais (PCA) é simples, não-paramétrica
e nos ajuda a extrair informações de conjuntos de dados com muitas variáveis,
reduzindo a dimensionalidade no sentido de encontrar quais são os elementos com
maior variâncias no seu conjunto de dados. No caso de um espectro, temos uma
medida de fluxo para cada intervalo em comprimento de onda. No caso de um cubo
de spaxels, temos as dimensões espaciais também, obtendo assim uma infinidade de
variáveis. O grande problema do PCA é que você tem a resposta, mas não sabe a
pergunta.

Hoje, PCA é utilizado exaustivamente em várias áreas de conhecimento. \ojo
\citneed \textcolor{blue}{Referências e exemplos de uso do PCA, mais pra
frente procuro}.
% http://arxiv.org/abs/1009.4974
% http://arxiv.org/abs/1208.3700
% http://arxiv.org/abs/astro-ph/0010424
% http://arxiv.org/abs/1310.4217

Com a criação da técnica de Tomografia PCA citada anteriormente temos, além da
análise das componentes principais, uma imagem formada pela matriz de
covariância (as questões matemáticas serão abordadas no
Capítulo \ref{sec:PCAeTomoPCA}). Assim podemos saber o peso, ou relevância, de
cada componente principal (daqui pra frente CP) relacionada a uma posição na
galáxia, aliando a infinidade de informações físicas presentes nos espectros e
nas imagens.

\section{Este trabalho}
\label{sec:Intro:ThisWork}

Através da colaboração do Grupo de Astrofísica da Universidade Federal de Santa
Catarina (GAS-UFSC) com o grupo de pesquisadores do projeto CALIFA tive a
oportunidade de trabalhar com os dados de IFS das galáxias observadas por esse
projeto, que ainda está em andamento. O seu primeiro {\em Data
Release\footnote{\url{http://www.caha.es/CALIFA/public_html/?q=content/califa-dr1}}}
\citep{Husemann2013} possui $100$ objetos e por volta de $400$ mil espectros. A
previsão é que ao término do projeto serão observadas até $600$ objetos. Embora
outros surveys de IFS estão completos ou em andamento (ver Seção
\ref{sec:CALePyC:Apresent}), o CALIFA é o que podemos chamar de {\em estado da
arte} em surveys de espectroscopia de campo integrado (IFS).

Dado o grande número de informações sobre cada galáxia necessitamos de um {\em
pipeline} que faça a organização de todos os dados para que cálculos e gráficos
de mais variadas dificuldade sejam fácil para que até um programador de nível
iniciante possa fazer. André L. de Amorim, colaborador de nosso grupo,
juntamente com outros colaboradores do projeto CALIFA construiu o PyCASSO ({\em
Python CALIFA \starlight Synthesis Organizer}) \citep{CidFernandes2013a} que
faz a organização dos dados que vêm do {\em survey} juntamente com a síntese de
populações estelares feitas com o \starlight, facilitando, e muito, o trabalho
de quem usa estes dados. Sem a ajuda deste organizador, este trabalho seria
muito mais difícil e acredito que não sería realizável em tempo hábil.

\subsection{Organização deste trabalho}

No seguinte capítulo faço uma apresentação mais detalhada do {\em survey}
CALIFA, bem como do {\em pipeline} PyCASSO. Discuto sua a utilização com alguns
exemplos de uso.

O terceiro capítulo descreve matemáticamente a técnica PCA e a Tomografia PCA,
bem como sua utilização no presente trabalho. 

A partir do quarto capítulo descrevo e discuto os mais diferentes {\em
sabores}\footnote{sabores aqui é usado apenas como uma maneira de tipificar
diferentes pré-processamentos dos dados antes da execução da análise de
componentes principais.} de execução do PCA, juntamente com suas implicações aos
resultados das análises. Juntamente apresento diversos gráficos demonstrando as
diferenças entre a utilização de cada pré-processamento juntamente com a análise
da Tomografia PCA e das CPs, bem como as principais referências na área.

Com todo o arcabouço teórico em mãos, no capítulo cinco faço um estudo de caso
para $9$ objetos do survey CALIFA, escolhidos por seu tipo morfológico. São $3$
galáxias espirais, $3$ elípticas e $3$ objetos compostos (mergers). Faço também
uma espécie de engenharia reversa através de correlações e comparações com os
resultados da síntese de populações estelares executadas pelo \starlight nesses
objetos.

Por fim, apresento as conclusões e as perspectivas futuras deste trabalho e da
nossa colaboração com o projeto CALIFA no sexto e último capítulo. Devido ao
grande número de imagens e dados para análise, muitas delas ficaram em anexo,
bem como um ou outro código Python usado.

% End of this chapter
