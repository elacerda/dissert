%%%%%%%%%%%%%%%%%%%%%%%%%%%%%%%%%%%%%%%%%%%%%%%%%%%%%%%%%%%%%%%%%
% Dissertacao de Mestrado / Dept Fisica, CFM, UFSC              %
% Andre@UFSC - 2011                                             %
%%%%%%%%%%%%%%%%%%%%%%%%%%%%%%%%%%%%%%%%%%%%%%%%%%%%%%%%%%%%%%%%%

%:::::::::::::::::::::::::::::::::::::::::::::::::::::::::::::::%
%                                                               %
%                          Capítulo 2                           %
%                                                               %
%:::::::::::::::::::::::::::::::::::::::::::::::::::::::::::::::%

%***************************************************************%
%                                                               %
%                            Galex                              %
%                                                               %
%***************************************************************%

\chapter{O {\em Galaxy Evolution Explorer} (\galex)}
\label{sec:Galex}


%***************************************************************%
%                                                               %
%                     Galex - Objetivos                         %
%                                                               %
%***************************************************************%

\section{Objetivos do \galex}
\label{sec:Galex:Objetivos}

O {\em Galaxy Evolution Explorer} (\galex) é um telescópio espacial de pequeno
porte (espelho primário com $50\,\mathrm{cm}$) da NASA\footnote{{\em NASA Small
Explorer} ({\em SMEX}) - \url{http://explorers.gsfc.nasa.gov/missions.html}},
lançado em 28 de abril de 2003 para conduzir um {\em survey} de todo o céu numa
faixa espectral do UV, entre $1350$ e $2750\,\text{\AA}$. O objetivo principal
do \galex é estudar a evolução da taxa de formação estelar em galáxias
\citep{Martin2005}. Os dados coletados pela missão são publicados em {\em Data
Releases} periódicos, denominados {\em General Releases}. Este trabalho foi
realizado sobre os dados do sexto {\em General Release}, GR6.

A missão consiste em uma série de {\em surveys} fotométricos e espectroscópicos
(ver tabela \ref{tab:GalexSurveys}). Destes, os principais {\em surveys} são o
{\em All Sky Survey} (AIS), cobrindo todo o céu numa profundidade menor, e o
{\em Medium Imaging Survey} (MIS), cobrindo uma área limitada porém mais
profunda, que foram utilizados neste trabalho. O imageamento é feito em duas
bandas espectrais: ultravioleta distante ({\em far ultraviolet}, FUV), de $1350$
a $1750\,\text{\AA}$, e ultravioleta próximo ({\em near ultraviolet}, NUV), de
$1750$ a $2750\,\text{\AA}$. As curvas de transmissão dos filtros utilizados
nessas bandas podem ser vistos na Figura \ref{fig:GalexFilters}. A
espectroscopia é feita inserindo-se no caminho ótico um grisma, que consiste num
prisma combinado com uma rede de difração. Obtém-se deste modo um espectro de
baixa resolução para cada objeto na imagem, conforme descrito por
\citet{Morrissey2007}.

\begin{table}
	\caption[{\em Surveys} realizados pelo \galex.]{{\em Surveys} realizados pelo
	\galex. O CAI consiste em observações de anãs brancas para calibração. A
	cobertura do céu é dada em graus quadrados. No caso do NGS, a magnitude limite
	é dada em unidades de densidade superficial de magnitude
	[$\mathrm{arcsec}^{-2}$]. Informações retiradas de \citet{Martin2005}.}
	\begin{tabular}{l r r}
		{\em Survey} & Cobertura [$\mathrm{graus}^2$] & Mag. AB limite \\ 
		\midrule
		{\em Calibration Imaging (CAI)}              &         - &            - \\
		{\em All-sky Imaging Survey (AIS)}           & $26\,000$ &       $20.5$ \\
		{\em Medium Imaging Survey (MIS)}            &  $1\,000$ &         $23$ \\
		{\em Deep Imaging Survey (DIS)}              &      $80$ &         $25$ \\
		{\em Nearby Galaxy Survey (NGS)}             &      $80$ &       $27.5$ \\
		{\em Wide Field Spectroscopic Survey (WSS)}  &      $80$ &         $20$ \\
		{\em Medium-deep Spectroscopic Survey (MSS)} &       $8$ & $21.5$--$23$ \\
		{\em Deep Spectroscopic Survey (DSS)}        &       $2$ &   $23$--$24$ \\
	\end{tabular}
	\label{tab:GalexSurveys}
\end{table}

\begin{figure}
	\includegraphics{figuras/galex-filters.eps}
	\caption[Curvas de transmissão dos filtros do \galex.]
	{Curvas de transmissão dos filtros do \galex, medidas em
	laboratório, representadas por linhas sólidas \citep{Morrissey2005}. Os
	comprimentos de onda efetivos dos filtros são $\lambda_{FUV}=1528\text{\AA}$ e
	$\lambda_{NUV}=2271\text{\AA}$. Para comparação é mostrada a curva eficiência
	para os filtros do \SDSS (linhas tracejadas). Dados retirados do {\em
	website} mantido por Peter Capak:
	\url{http://www.astro.caltech.edu/~capak/cosmos/filters/}}
	\label{fig:GalexFilters}
\end{figure}

Os {\em surveys} do \galex foram planejados de forma a se valer de outros {\em
surveys} já existentes em outros comprimentos de onda. A Figura
\ref{fig:GalexSDSSOverlap} mostra a sobreposição da área observada\footnote{{\em
Footprint}, no linguajar astronômico.} pelos {\em surveys} AIS e MIS do \galex e
do {\em Sloan Digital Sky Survey} (\SDSS). O objetivo primário da missão do
\galex é calibrar da taxa de formação estelar no universo local e determinar o
histórico cosmológico de formação estelar entre os {\em redshifts} $0 < z < 2$
\citep{Martin2005}. A comparação com dados de {\em surveys} em outros
comprimentos de onda tem um papel fundamental no cumprimento deste objetivo.

\begin{figure}
	\includegraphics[width=1.0\columnwidth]{figuras/footprint.eps}
	\caption[{\em Footprint} dos {\em surveys} \galex AIS, MIS e SDSS]
	{{\em Footprint} dos {\em surveys} \galex GR2+3 AIS (azul), MIS (vermelho) e
	SDSS DR6 (verde), de \citet{Budavari2009}.}
	\label{fig:GalexSDSSOverlap}
\end{figure}


%***************************************************************%
%                                                               %
%                Galex - O céu no ultravioleta                  %
%                                                               %
%***************************************************************%

\section{Histórico do estudo do céu no UV}
\label{sec:Galex:CeuUV}

Devido à alta absorção da luz na banda UV causada pela camada de ozônio,
observações do céu na banda UV precisam ser feitas fora da atmosfera terrestre.
Não é de se estranhar, portanto,  que o trabalho nesta faixa espectral tenha
progredido menos do que na faixa do óptico e do infravermelho.

O primeiro trabalho sistemático de observação em UV foi feito pelo {\em Orbiting
Astronomical Observatory 2} \citep{Code1970}, obtendo fotometria e
espectroscopia de estrelas brilhantes, aglomerados globulares e galáxias
próximas. Durante as décadas de 1970 e 1980, este e outros satélites como o TD-1
\citep{Boksenberg1973}, o {\em Astronomical Netherlands Satellite}
\citep{vanDuinen1975} e o {\em International Ultraviolet Explorer}
\citep{Kondo1987} -- o primeiro satélite a utilizar um detetor de imageamento UV
-- forneceram os dados fundamentais para os modelos de síntese de população
estelar de galáxias. {\em Surveys} de campo amplo foram feitos por uma câmera
lunar erguida por astronautas da {\em Apollo 16} \citep{Carruthers1973}, a bordo
do {\em Skylab} \citep{Henize1975} e pelo instrumento {\em FAUST} a bordo do
{\em Spacelab} \citep{Bowyer1993}. Muitas imagens UV também foram obtidas pelo
{\em Ultraviolet Imaging Telescope} em duas missões em ônibus espacial
\citep{Stecher1997}.


%***************************************************************%
%                                                               %
%                     Galex - Resultados obtidos                %
%                                                               %
%***************************************************************%

\section{Resultados obtidos pelo \galex}
\label{sec:Galex:Resultados}

O \galex fez o primeiro {\em survey} do céu inteiro em UV. As regiões próximas
ao plano da Galáxia foram evitadas para não danificar os detetores. Pode-se ter
uma ideia do sucesso desta missão considerando a grande quantidade de artigos
publicados\footnote{Há uma lista com as mais de 500 publicações relacionadas ao
projeto do \galex em
\url{http://www.galex.caltech.edu/researcher/publications.html}}. Abaixo segue
um resumo dos resultados mais notáveis.

\citet{Wyder2007} analisaram a distribuição de galáxias em função da cor UV e da
magnitude absoluta no universo local. Esta distribuição é conhecida como {\em
Diagrama Cor--Magnitude} (CMD, na sigla em inglês para {\em Color--Magnitude
Diagram}). Os autores usam {\em redshifts} e fotometria óptica obtidas do \SDSS
junto com fotometria UV do {\em survey} MIS do \galex. A amostra do \SDSS é
correlacionada com a do \galex procurando o objeto do \galex mais próximo de
cada objeto \SDSS até um limite de $4$ segundos de arco.

\begin{figure}
	\includegraphics[width=0.5\columnwidth]{figuras/cmd-wyder.eps}
	\caption[Diagrama cor--margnitude em ultravioleta.]
	{Diagrama cor--margnitude em ultravioleta. \citep[figura 7]{Wyder2007}.}
	\label{fig:WyderCMD}
\end{figure}

O diagrama cor--magnitude ($\mathrm{NUV}-r$ contra $M_r$) elaborado por
\citeauthor{Wyder2007} mostra a separação das galáxias nas sequências azul e
vermelha (Figura \ref{fig:WyderCMD}). Esta distribuição bimodal é um resultado
bem conhecido na astronomia \citep{Baldry2004}. Porém, diferente do diagrama
cor--magnitude para a faixa espectral do óptico, a distribuição de cores em UV
não pode ser ajustada somente pela soma de duas gaussianas, há um excesso de
objetos nas cores intermediárias entre os picos azul e vermelho. A boa separação
entre as sequências é atribuída a uma maior sensibilidade à formação estelar
recente.

\citet{Martin2007} investigaram as propriedades das galáxias entre as sequências
vermelha e azul para a mesma amostra citada acima. As galáxias nesta região
intermediária são preferencialmente galáxias com núcleo ativo ({\em Active
Galactic Nucleus}, AGN). Os autores estimam o fluxo de massa de galáxias indo da
sequência azul para a vermelha.

Ainda para a mesma amostra, \citet{Schiminovich2007} investigaram a correlação
entre a morfologia das galáxias e a sua posição no CMD. A função de luminosidade
UV do universo local é medida -- pela primeira vez, segundo os autores -- com
relação aos parâmetros estruturais e à inclinação das galáxias.

A missão do \galex se encerrou em 31 de dezembro de 2011. Dados coletados após o
GR6, como as observações no mesmo campo utilizado na missão {\em Kepler},
observações de M31 e da Nuvem de Magalhães, entre outros, serão liberados num
último {\em data release}, GR7. Os dados obtidos pelo \galex permanecerão
disponíveis publicamente no {\em Multi-Mission archive at the Space Telescope
Science Institute} (MAST).



%***************************************************************%
%                                                               %
%                     Galex - Banco de dados                    %
%                                                               %
%***************************************************************%

\section{Data releases e banco de dados}
\label{sec:Galex:BancoDeDados}

A liberação dos dados do \galex é feita anualmente em {\em General Releases}
(GR). Os dados consistem basicamente em imagens e catálogos, divididos em campos
({\em tiles}) com área de aproximadamente $1,2$ graus quadrados. Devido ao modo
como o \galex faz as observações, um determinado objeto pode estar presente em
mais de um campo. A tabela \ref{tab:GalexReleases} mostra o número cumulativo de
campos observados por {\em survey} em cada GR\footnote{Informações retiradas do
{\em website} do GR6: \url{http://galex.stsci.edu/GR6/}}. Observações de
pesquisadores convidados ({\em Guest Investigators}, GI) foram selecionadas de
forma a complementar os {\em surveys}.

\begin{table}
	\caption[Campos observados em cada {\em General Release} do \galex.]{Campos
	observados em cada {\em General Release} do \galex.}
	\begin{tabular}{l r r r r r r r r}
		{\em Release} & AIS       & DIS   & MIS      & NGS   &     GI   &  CAI & Espectros & Total     \\
		\midrule
		GR1           &  $3\,074$ &  $14$ &    $112$ &  $52$ &        - &    - &       $7$ &  $3\,259$ \\
		GR2/GR3       & $15\,721$ & $165$ & $1\,017$ & $296$ &    $288$ & $20$ &      $41$ & $17\,548$ \\
		GR4/GR5       & $28\,269$ & $292$ & $2\,161$ & $458$ &    $788$ & $38$ &     $174$ & $32\,180$ \\
		GR6           & $28\,889$ & $338$ & $3\,479$ & $480$ & $1\,314$ & $51$ &         - & $34\,551$ \\
	\end{tabular}
	\label{tab:GalexReleases}
\end{table}

Para facilitar o acesso aos dados do \galex, o MAST desenvolveu uma ferramenta
chamada {\em GalexView} (Figura \ref{fig:GalexView}), utilizando tecnologia {\em
Adobe Flex}\footnote{{\em Adobe Flex} é um {\em framework} de código aberto que
permite desenvolver aplicações para {\em web browsers}. Ver
\url{http://www.adobe.com/products/flex.html}.}. Desta forma o {\em GalexView }
pode ser acessado através de seu {\em website}\footnote{GalexView:
\url{http://galex.stsci.edu/GalexView/}} em qualquer {\em web browser} que tenha
suporte ao {\em Adobe Flash Player}\footnote{{\em Adobe Flash Player} é uma
extensão multiplataforma para {\em web browsers} que provê capacidade de
visualização de conteúdo {\em flash} gerado tanto pelos seus editores
proprietários quanto por ferramentas de terceiros. Ver
\url{http://www.adobe.com/products/flashplayer/}.}.

\begin{figure}
	\includegraphics[width=0.7\columnwidth]{figuras/galexview.eps}
	\caption[Tela do programa{\em GalexView}.]
	{Tela do programa {\em GalexView}, com a visualização da galáxia M101.}
	\label{fig:GalexView}
\end{figure}

Através do {\em GalexView} é possível fazer buscas, visualizar e obter imagens e
catálogos dos campos do \galex. As buscas podem ser feitas de forma bastante
versátil, tanto pelo nome do objeto quanto pelas coordenadas do céu. O formato
de entrada é flexível o suficiente para evitar os problemas causados por
idiossincrasias na notação de coordenadas (por exemplo, tanto ``14h03m12.6s
+54d20m56.7s'' quanto ``14 03 12.6 54 20 56.7'' ou ``210.83 54.35'' apontam para
a mesma região). A sua interface permite filtrar o conteúdo retornado pelas
buscas, separando por {\em surveys}. Há também uma ferramenta de histograma,
permitindo filtrar pelos valores das colunas dos catálogos. Os objetos
selecionados na busca aparecem marcados na visualização da imagem. Utilizando um
sistema do tipo ``carrinho de compras'', pode-se selecionar campos e objetos de
interesse, para, ao final do uso do sistema, baixar toda a seleção de uma vez.

Tanto o {\em GalexView} quanto outras ferramentas de busca do MAST, como o
\galex {\em Search Form} e o \galex {\em Tilelist}, são construídos sobre um
{\em banco de dados relacional} acessado através da linguagem {\em SQL}
\citep{Chamberlin1974}. Muito comum na indústria, bancos de dados relacionais
dispõem em geral de uma vasta gama de ferramentas para gerenciamento dos dados.
Uma de suas grandes vantagens é o uso de índices\footnote{Um índice numa tabela
de banco de dados é uma estrutura que copia partes da tabela numa determinada
ordem, de forma a aumentar a velocidade de acesso aos dados ao custo de espaço
de armazenamento.} para agilizar o acesso a dados. Embora a tecnologia exista
desde a década de 1970 \citep{Codd1970}, até uma década atrás suas vantagens
eram praticamente neglicenciadas na astronomia.

Bancos de dados relacionais e ferramentas para gerenciamento e acesso a dados
serão tratados com mais detalhes no capítulo \ref{sec:Crossmatch}.


% End of this chapter
