%%%%%%%%%%%%%%%%%%%%%%%%%%%%%%%%%%%%%%%%%%%%%%%%%%%%%%%%%%%%%%%%%
% Dissertacao de Mestrado / Dept Fisica, CFM, UFSC              %
% Andre@UFSC - 2011                                             %
%%%%%%%%%%%%%%%%%%%%%%%%%%%%%%%%%%%%%%%%%%%%%%%%%%%%%%%%%%%%%%%%%

%***************************************************************%
%                                                               %
%                          Resumo                               %
%                                                               %
%***************************************************************%

\begin{abstract}[Resumo]

Levantamentos espectroscópicos de galáxias, como o \SDSS, estão sendo levados para um novo nível através das unidades
de campo integral (IFU), transformando o foco das propriedades das galáxias como um todo para a física interna das
galáxias. O {\em survey} CALIFA é pioneiro nesta nova geração, proporcionando espectroscopia espacialmente resolvida para
centenas de galáxias de todas as formas e massas. Vários estudos têm sido realizados com os cubos de dados deste {\em
survey}, examinando as distribuições espaciais de linhas de emissão, populações estelares, cinemática do gás e das
estrelas, etc. Neste trabalho exploramos a análise dos cubos do CALIFA através da perspectiva
matemática da técnica de tomografia da análise de componentes principais (PCA) criada por Steiner e colaboradores.
Utilizamos a tomografia PCA em 8 galáxias do CALIFA (4 {\em late-types}, 2 {\em early-types} e 2 {\em mergers}). Linhas
de emissão são mascaradas em nossas análises para que nos concentremos nas propriedades de populações estelares.

As escalas físicas e angulares abarcadas pelos nossos cubos de dados (cobrindo galáxias inteiras) são muito maiores do
que aquelas cobertas por aplicações anteriores desta técnica. Essa diferença nos impeliu a introduzir uma simples
variação no estágio pré-PCA, normalizando os espectros a uma escala comum de fluxo, destacando variância espectral
qualitativa (em oposição à quantitativa relacionada com a amplitude). Examinamos também cubos sintéticos obtidos com
ajustes espectrais dos dados originais. Isto elimina o ruído e efeitos de calibração de fluxo. É bem sabido que não
podemos atribuir sentido físico os resultados da PCA. Para resolver este problema, utilizamos uma análise de populações
estrelares feitas com o \starlight sobre os mesmos dados para obtermos, através de uma engenharia reversa, o significado
astrofísico das componentes principais.

Em cada galáxia fomos capazes de encontrar alguma correlação entre componentes principais e parâmetros físicos. Nós
detectamos que a primeira componente principal em galáxias espirais geralmente se correlaciona com a idade estelar média
(\meanL{\log\ t}). Em mergers a componente principal reflete essencialmente variações da extinção global ($A_V$). Em
alguns casos a tomografia PCA nos ajuda a localizar erros no cubos originais pois normalmente a grande variância imposta
por pixels danificados aparece em componentes principais isolados.
 
\end{abstract}

% End of resumo
